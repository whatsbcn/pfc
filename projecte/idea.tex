\chapter{Funcionalitats}

Un cop tenim una idea més clara del ``a què'' vol donar solució el projecte, ens cal entrar una mica més
detall en com ho vol arribar a fer. Per aquest motiu tot seguit es defineixen les diferents funcionalitats 
que ens ha d'oferir el nostre rootkit. \\

Tal i com ara veurem, les funcionalitats del rootkit, han estat separades segons els privilegis amb què s'executaria 
el rootkit.

\section{Nivells de privilegis}

En la majoria de sistemes operatius actuals, existeix una separació de privilegis entre usuaris. Hi
han diferents nivells d'usuari, tant usuaris amb molt pocs privilegis, com usuaris amb tots els privilegis
possibles. Els usuaris que disposen del màxim nivell de privilegis, s'anomenen usuaris administradors. \\
Aquests usuaris, acostumen a no tenir limitacions alhora de llançar accions a la màquina, és per aquest
motiu, que l'usuari preferit per un atacant, és l'usuari administrador. \\

\section{Mode no privilegiat}

Aquestes són les funcionalitats que ens ofereix el rootkit quan s'executa com a \mbox{usuari no privilegiat}
\footnote{Tot i que el títol de la secció s'anomena mode no privilegiat, aquest nom no fa referència al mode d'execució
en què els nuclis dels sistemes operatius executen el codi d'usuari} del sistema.

\subsection{Executable ELF estàtic}
Per tal de fer el més portable possible el nostre rootkit, i poder així ser executat en gairebé qualsevol sistema, ens 
interessa que aquest sigui estàtic. Això vol dir que el nostre rootkit incorporarà tot el codi necessari per tal de portar 
a terme totes les seves funcions i per tant podrà ser executat independentment de les llibreries i versions que es trobin a
les màquines.

\subsection{Multiplataforma i multiarquitectura}
Tot i que el rootkit estigui basat per a ser executat en sistemes UNIX que compleixin l'estandard POSIX, un mateix sistema 
operatiu pot estar compilat per a ser executat sobre un processador de 32 o 64 bits, així com una arquitectura diferent
(intel, ARM, MIPS, etc). A part d'això, hi han moltes variants de sistema operatiu que provenen de UNIX com són  Linux, 
FreeBSD, NetBSD, Solaris, etc. És per aquest motiu, que la nostre intenció és que el rootkit suporti el reguitzell més gran 
possible de arquitectures i variants de UNIX.

\subsection{Connexió directa}
L'arquitectura de moltes aplicacions en xarxa avui en dia, és la de client-servidor, on el client estableix una connexió
amb el servidor, i a partir d'aquesta s'estableix una comunicació. Aquesta arquitectura ha de ser oferta pel nostre rootkit. 
El rootkit ha de ser capaç d'obrir un port a la màquina, i quedar escoltant a l'espera de què el seu propietari s'hi connecti,
i així oferir-li un accés a la màquina.

\subsection{Obtenció d'una shell i un TTY}
A part del fet de permetre'ns la connexió, és molt important el típus de connexió que ens permet el rootkit. El més còmode, és que ens
ofereixi una connexió a una shell tipus bash. A més, si ens la ofereix a través d'un TTY, la podrem utilitzar juntament totes les eines
que ofereix l'interpret de comandes com poden ser editors, assistents, etc.

\subsection{Mode comanda / Mode servei}
Tot i que la majoria de vegades ens acabarà interessant deixar el nostre rootkit corrent a la màquina com a servei, no sempre
és la funcionalitat que voldrem. Tal i com hem comentat en la definició del problema, quan estiguem en mig d'una intrusió, ens interessarà
executar comandes còmodament i per tant el rootkit ens ha d'ajudar en aquest moment concret. En aquest instant ja ens interessarà 
disposar d'una shell, transferir fitxers, etc, però només per acabar de realitzar la intrusió.

La funcionalitat que es vol mostrar en aquest punt, és la de poder utilitzar el rootkit com una comanda i no com a servei.

\subsection{Transferència de fitxers}
De la mateixa manera que ens interessa obtenir una shell a la màquina on tenim instal·lat el rootkit, també ens interessa
poder tenir total control sobre el sistema de fitxers, i per tant la possibilitat de pujar o descarregar fàcilment 
qualsevol fitxer que es trobi o que necessitem al disc.

\subsection{Comunicació xifrada}
Tota la comunicació entre la part servidor i la part client del rootkit, es fà a través de la xarxa. Per tal d'ocultar al màxim 
tota aquesta comunicació i fer-la de la manera més segura possible, el nostre rootkit ha d'implementar algun algoritme de xifratge.

\subsection{Autenticació per contrasenya}
Per tal de permetre que un usuari faci servir el rootkit o no, aquest està protegit amb una contrasenya que és introduïda en el moment 
de la seva compilació.

\subsection{Detecció del rootkit}
Una funcionalitat que ens pot interessar molt, és la de detectar si una màquina té instal·lat el rootkit tot i no saber-ne la contrasenya. 
D'aquesta manera una màquina que hàgim infectat fà molt de temps, podrem saber si encara hi ha el nostre rootkit instal·lat d'una manera molt 
senzilla i sense arribar a llançar cap nou procés.

\subsection{Proteccions de l'executable}
Ens interessa protegir l'executable per tal que algú que busqui intencionadament entendre quin és el seu funcionament,
ho tingui el màxim de difícil possible. És per aquest motiu, que el rootkit incorpora tècniques anti desensamblat i anti depuració.

\subsection{Supervivència del rootkit}
Un cop instal·lat el nostre rootkit, voldrem que cada vegada que la màquina reinicii, aquest es torni a executar. 
Aconseguir això ens serà més fàcil si el propi rootkit no permet múltiples execucions. El millor serà que en comptes
de intentar llançar-lo només una vegada, ell mateix detecti que està en execució i en cas de estar-ho, acabi l'execució. \\

Si aconseguim això, podrem fer que el rootkit es llanci en diferents moments de l'arrancada, per tal que cas que algun dels mètode 
d'arrancada falla, tindrem moltes probabilitats que el rootkit segueixi sent executat al arrancar. \\

També ens interessarà disposar de més d'un mètode per rearancar el rootkit.

\subsection{Tasques programades}
És molt comú en entorns UNIX utilitzar el servei de cron per a realitzar tasques programades a hores o dies concrets. Ens interessa
poder executar tasques en la màquina infectada sense que l'administrador de la màquina se n'adoni, per tant ens anirà molt bé que 
el nostre rootkit ens implementi aquesta funcionalitat.

\subsection{Ocultació}
Com portem dient des del principi, un dels seus objectius principals, és estar ocult als ulls de l'administrador del sistema. Per aquest motiu el 
nostre rootkit, ha de perdurar el màxim ocult possible. Hem de intentar que no cridi gens l'atenció.

\subsection{Heartbeat}
Ens interessa que el rootkit ens estigui dient "constantment" que està actiu. D'aquesta manera podrem tenir un control de les diferents
màquines que tenim infectades, i si en algun moment, en perdem alguna d'elles.

\subsection{Independència de la shell}
Per tal de tenir el mateix intèrpret de comandes independentment del sistema, i evitar algun sistema de loggeig que acostumen a 
incorporar per defecte, ens interessa incorporar dintre el propi rootkit la shell.

\section{Mode administrador}

\subsection{Connexió inversa}
De la mateixa manera que en la connexió directa, el rootkit ha de permetre establir una connexió al seu propietari per tal de connectar-se
a la màquina, però aquest cop, aquesta connexió no és iniciada des de la part client, sinó que és el propi rootkit que es connecta 
a la part client. D'aquesta manera, en sistemes on no estan permeses les connexions d'entrada a qualsevol port d'una màquina, però si
que ho estan les de sortida, el nostre rootkit ens permetrà connectar-nos còmodament.

\subsection{Tècniques per evitar firewalls i filtres}
Per tal de poder utilitzar el rootkit en configuracions de xarxa molt restrictives, aquest ha d'implementar diferents modes de connexió.
Aquests seran implementats a través d'un raw socket, i per tant caldrà executar-lo en mode privilegiat per fer-ne ús.

\subsection{Keylogger}
Molts dels cops que obtinguem accés de root, probablement serà a través d'algun bug. És per aquest motiu, que ens pot interessar molt
obtenir el password de root o d'altres usuaris vàlids del sistema. Aquests passwords els podrem obtenir en el moment que algun usuari
els escrigui en un teclat físic, o a través d'un pty gracies al keylogger que implementarà el rootkit.

\subsection{Injecció de codi en memòria del nucli}
Com s'ha comentat abans, els usuaris administradors, acostumen a tenir accés total a la màquina, i a poder realitzar gairebé qualsevol tasca.
Una de les coses que pot fer l'usuari administrador en un sistema Linux, és llegir i escrire directament a una posició de la
memòria del sistema, d'aquesta manera es pot arribar a modificar la part de memòria que utilitza el sistema operatiu per funcionar,
obtenint així un control total sobre què mostrar al sistema operatiu i podent ocultar tant com es vulgui el nostre rootkit.
Ens interessa que el nostre rootkit injecti codi en memòria del nucli per ocultar-se el màxim possible.




