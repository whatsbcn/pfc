\chapter{Conclusions}

En aquest capítol, s'intentarà analitzar fins a quin punt s'han pogut complir els objectius que es varen 
establir a l'inici del projecte. Es repassaran les diferents funcionalitats que es volien implementar 
i es comentaran quines no han estat possibles d'implementar, i quines s'han afegit i perquè. Finalment
s'exposen futures línies de treball per a millorar el rootkit i es conclou resumint com ha estat 
l'experiència. \\

\section{Objectius generals}

A part de la creació en sí del propi rootkit, els objectius generals que ens havíem marcat eren els 
d'aconseguir que el rootkit perdurés ocult, ens proporcionés un accés remot a la màquina, ens permetés 
recuperar el nivell de privilegi, ens permetés administrar la màquina, que sobrevisqués a possibles 
canvis d'estat de la màquina, com poden ser actualitzacions i reiniciades del sistema, així com que 
ens ajudés a elevar privilegis i a consolidar els actuals. \\

Considero que tots aquests objectius han estat complerts, ja que s'han desenvolupat funcionalitats 
per cadascun d'ells.  \\

\section{Funcionalitats}

Pel què fa a les funcionalitats, s'han acabat implementant les següents:

\begin{itemize}
	\item Executable ELF estàtic
	\item Multiplataforma i multiarquitectura
	\item Connexió directa
	\item Obtenció d'una shell i un TTY
	\item Mode comanda / Mode servei
	\item Transferència de fitxers
	\item Comunicació xifrada
	\item Autenticació per contrasenya
	\item Detecció del rootkit
	\item Proteccions de l'executable
	\item Supervivència del rootkit
	\item Tasques programades
	\item Ocultació
	\item Heartbeat
	\item Independència de la shell
	\item Proxy socks
	\item Connexió inversa
	\item Tècniques per evitar firewalls i filtres
	\item Keylogger
	\item Injecció de codi en memòria del nucli
\end{itemize}

En definitiva s'han implementat totes les funcionalitats del plantejament inicial menys d'injecció de codi 
en memòria del kernel, que no ha pogut ser implementada per causes externes. \\

Com s'ha comentat al seu punt, en les versions actuals del kernel de linux, ja no és possible llegir 
ni escriure directament a la memòria a través d'un dispositiu. I com que un dels punts clau d'aquest 
rootkit, era que fos per sistemes actuals, aquesta funcionalitat va acabar sent descartada. \\

Pel què fa a les funcionalitats que es varen afegir per tal de suplir aquesta funcionalitat no implementada,
varen ser: 
\begin{itemize}
	\item Mode de comunicació RAW
	\item Servei socks4 i socks4a
\end{itemize}

\section{Futures línies de treball}

Per tal de millorar l'èina desenvolupada, penso que els principals camins a seguir partint de la 
implementació actual, serien:

\begin{description}
	\item[Més serveis suportats pel keylogger] Més capacitat de capturar passwords. Ens interessa que el keylogger sigui capaç
		de funcionar contra servidors web, de correu, imap, pop, etc.
	\item[Reimplementació de la shell] Reimplementació des de zero de la shell interna del rootkit. D'aquesta manera obtindríem
		una shell més petita que a més, només tindria les funcionalitats necessàries.  
	\item[Túnels] Implementació de túnels entre el client i el launcher. Això podria permetre coses com que el 
		servidor de socks fos accessible des d'un túnel establert pel client.
	\item[Infecció de binaris] Implementació d'alguna tècnica per a infectar binàris. D'aquesta manera no caldria afegir cap 
		fitxer al sistema, sinó que afegint alguns bytes en algun executable de l'arranc, seria suficient
		per a que el rootkit perdurés entre reiniciades.
	\item[Neteja del rastre] Implementació d'un mòdul per a la neteja de logs i el possible rastre que s'hagi pogut deixar
		alhora de realitzar la intrusió.
\end{description}

Espero poder arribar a implementar totes aquests funcionalitats en un futur no molt llunyà.

\section{Opinió personal}

Personalment estic molt content de com ha anat tot el projecte. Des de la relació que he tingut amb el
meu tutor, passant per l'ús d'èines com el \LaTeX  o la innovació i recerca que he hagut de realitzar amb aquest 
projecte, fins a l'assoliment dels objectius plantejats. \\

També penso que hi han punts d'aquest projecte que seran difícilment puntuats o tinguts en compte com 
són l'estabilitat i l'usabilitat que se li ha arribat a donar al rootkit. Dic que és difícil de puntuar, 
perquè en un projecte com aquest, són molts els detalls tècnics que no s'han pogut acabar comentant, i
que per part del jurat és difícil entrar-hi tant a fons com per a poder-los valorar. \\

Considero que aquest projecte engloba almenys tres parts que per si soles són força importants
tant la seva component de innovació i recerca, com per la complexitat que
tenen. \\
Aquestes tres parts que han estat desenvolupades en el marc del projecte són:
\begin{itemize}
	\item El protocol de comunicació RAW
	\item El keylogger per OpenSSH
	\item La utilitat skpd 
\end{itemize}

En resum, considero que el projecte ha estat un èxit.

