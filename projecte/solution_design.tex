\chapter{Disseny de la solució}

Per tal de complir els nostres objectius comentats en els punts anteriors, s'ha hagut de fer
un gran treball de planificació i disseny. En aquest capítol, s'intenta mostrar i justificar 
les principals decisions de disseny.

\section{Disseny general del rootkit}

L'arquitectura del rootkit és la de client-servidor. Això significa que hi ha una part que 
s'executa a una màquina que ofereix ``serveis'' (el servidor), i una part que sol·licita i
rep els serveix que ofereix l'altre. \\

A partir d'aquest moment anomenarem ``launcher'' a la part servidor del rootkit, i ``client''
a la part client. \\

En un cas típic, el launcher serà la part del rootkit que s'executarà a la màquina que haguem 
compromès, i el client serà la part que executarà l'atacant per tal de connectar-se al launcher. \\

ESQUEMA AMB EL QUÈ ENS BASAREM A PARTIR D'ARA ON ES VEGIN AMBDUES PARTS \\

Com a tot projecte de software, el disseny és una de les parts més importants, i per començar 
cal decidir quines característiques volem assolir. En aquest cas s'ha escollit la portabilitat, 
extensibilitat, llegibilitat i la no repetició de codi, com a característiques principals del 
nostre disseny. \\

Tot seguit es mostra un esquema dels diferents components que formen el rootkit, tot descrivint 
la seva funcionalitat. \\

ESQUEMA DE MODULS DE LES DUES PARTS DEL ROOTKIT \\
% * config
%
%
%                                      __     __
%                                     |   sha1  |
%* launcher + dietlibc +  ucl  ---------- raw -----------  client + dietlibc
%     |                               |__ rc4 __|        
%     |                                 common
%     |
% antidebug
%    
%* keyloggerd
\subsection{config}

En moment de compilació, el rootkit ens demana una sèrie de preguntes per tal de configurar-se a gust
de l'usuari, i per tal d'adaptar-se millor a unes característiques o a unes altres. Entre aquestes
preguntes, hi ha el password a utilitzar en la part servidora, si activar el mode debug, etc. Aquest
mòdul no està incorporat en el rootkit, sinó que només serveix per a facilitar-ne la configuració.

\subsection{launcher}

Part nucli de la part servidor del rootkit. En aquesta part és on hi ha tota la estructura principal
de la part que s'instal·la a la màquina de la víctima.

\subsection{dietlibc}

Llibreria que ens permet compilar estàticament els executables, oferint-nos un tamany de fitxer
resultat, molt petit.

\subsection{ucl}

Aquesta llibreria és utilitzada per tal de comprimir l'executable tot afegint un wrapper per a què 
es descomprimeixi en moment d'execució. A part de oferir-nos un executable d'un tamany més reduït,
impossibilita l'anàlisi estàtic de l'executable.

\subsection{rc4}

Mòdul que ens permet xifrar la connexió utilitzant l'algoritme simètric rc4. Gràcies a aquest mòdul,
tota la informació que s'envia entre client i servidor, i client, és xifrada. 

\subsection{sha1}

Algoritme de hash utilitzat principalment per a obtenir un password d'una longitud fixa. En el moment
de compilació, el mòdul de configuració, sol·licita un password, aquest password haurà de ser especificat
per el client per tal de establir una connexió amb el launcher, i s'utilitzarà com a clau de xifratxe
de la comunicació.

\subsection{raw}

Mòdul que ens proporciona tota la funcionalitat del mode de funcionament raw. Tant la definició dels
paquets de xarxa, com les funcions dels diferents serveix necessaris.

\subsection{antidebug}

Mòdul que proveeix de les funcionalitats antidebug. Les diferents funcions que incorpora, són definides
com a inline, per tal que siguin incloses directament al codi.

\subsection{common}

Aquest mòdul incorpora les funcions genèriques més generals que utilitza el rootkit. És utilitzat tant per
el launcher, com per el client.

\section{Modes de funcionament}

Les funcionalitats que ens ha d'oferir el rootkit, varien segons els permisos que tinguem a la màquina on
l'estiguem executant, i de les nostres necessitats en cada moment. Per tal de poder escollir el mode 
en què volem llançar el rootkit, s'han separat en el què anomenem ``modes de funcionament''. 
Aquests modes defineixen les funcionalitats que podrem utilitzar i el com es podran utilitzar. Més endavant 
veurem que molt lligat a els modes de funcionament, tenim els modes de comunicació. Aquests també ens
permetran utilitzar unes funcionalitats o unes altres, i també dependran dels permisos que disposarem. \\

El mode d'execució del rootkit, ve definit pel nombre de paràmetres que se li passen en el moment de ser
executat. Cal tenir en compte que en cap moment el ``launcher'' ens mostrarà una ajuda de quins paràmetres 
es poden utilitzar, ni cap missatge d'error. D'aquesta manera es garanteix que si mai és descobert i analitzat, 
aquest no ofereix cap pista als possibles analitzadors. \\

En total el rootkit pot ser executat en tres modes: 

\begin{enumerate}
    \item Mode client
    \item Mode servidor no privilegiat
    \item Mode servidor privilegiat
\end{enumerate}

\subsection{Mode client} 

El mode client és un mode de funcionament en que per a ser utilitzat no és necessari cap permís especial. 
La idea d'aquest mode de funcionament, és llançar el rootkit, sense la intenció de tenir un servei corrent
a la màquina, sinó amb la intenció de disposar d'una funcionalitat en un moment concret. \\

En aquest mode, el launcher establirà una connexió TCP cap al client a la ip i port especificats per la línia
de comandes. Un cop establerta la comunicació, el client (que ha d'estar esperant la connexió del launcher), 
li transmetrà l'acció a executar, i aquest la portarà a terme. Un cop acabada l'acció, el launcher acabarà
i es desconnectarà del client. \\

ESQUEMA DEL MODE \\

Aquest mode de funcionament és molt interessant per el moment de la intrusió on ja som capaços d'executar
comandes a la màquina. Prèviament hem de ser capaços de deixar a la màquina l'executable del launcher, i
executar-lo en mode client. \\

Un cop fem això, el rootkit ens permetrà obtenir una shell enganxada a un TTY per tal de poder treballar 
còmodament, fer servir la màquina remota com a proxy SOCKS, així com pujar o baixar fitxers. \\

\subsection{Mode servidor no privilegiat}

Aquest mode de funcionament no requereix de més privilegis que el mode client, però en aquest cas, si que
ens interessarà tenir un servei executant-se a la màquina. \\

El fet de disposar d'un servei en constant execució o no, és la principal diferència entre el mode servidor
no privilegiat, i el mode client. En aquest mode, tindrem el launcher escoltant a un port TCP esperant que
el client es connecti per a sol·licitar una acció. \\

Aquest mode té l'inconvenient que en un entorn en què tinguem algun firewall, molt provablement, no ens 
servirà de res ja que les connexions cap al port del nostre launcher, molt probablement no estaran permeses. \\

ESQUEMA DEL MODE \\

\subsection{Mode servidor privilegiat}

El mode servidor privilegiat, és el mode que requereix de permisos d'administrador, i que alhora ens permetrà
fer ús de les característiques mes avançades del rootkit. \\

La principal diferència entre aquest mode de funcionament, és que al disposar de permisos d'administrador a
la màquina, podem realitzar tasques molt més avançades. Entre elles, tenim la possibilitat d'implementar 
sniffers a nivell d'aplicació per tal de poder capturar passwords, o el fet de poder utilitzar RAW sockets que
ens permetran utilitzar diferents modes de comunicació (se'n comenta el disseny més endavant), per tal de 
saltar-se la majoria de configuracions de firewall. \\

Per tots aquests motius, sempre que sigui possible ens interessarà utilitzar el rootkit en aquest mode. \\

ESQUEM DEL MODE \\

Tal i com podem apreciar en l'esquema, només en arrancar el launcher, estarem apunt per tal de rebre packets
del client, sol·licitant-nos efectuar alguna acció. \\

\subsubsection{Modes de comunicació}

Tal i com hem comentat anteriorment, els modes de comunicació depenen directament dels permisos que tindrem 
a la màquina. Principalment per aquest motiu, s'han fet dependre dels modes de funcionament, és a dir, segons el
mode de funcionament, podrem utilitzar uns modes de comunicació o uns altres. \\

El rootkit implementa diferents modes de comunicació, alguns ens permeten establir connexions totalment
ocultes i arribar a sobrepassar configuracions de xarxa molt restrictives. Poder utilitzar un protocol de 
comunicació o un altre, dependrà del mode de funcionament en en què haguem arrancat el rootkit. És per aquest
motiu, que el mode de comunicació anirà molt lligat al mode de funcionament. \\

En total tenim quatre modes de comunicació. \\

\begin{enumerate}
    \item TCP
    \item REV
    \item RAW
    \item LISTEN
\end{enumerate}

\subsection{TCP}

Aquest mode de comunicació, és utilitzat només en el cas en què el client vulgui connectar-se a un launcher que
s'ha executat en mode servidor no privilegiat. \\

El protocol de comunicació en aquest cas és: \\

ESQUEM DEL PROTOCOL \\

\begin{enumerate}
    \item El client estableix una connexió TCP amb a la màquina i port on està escoltant el launcher
    \item El client envia un paquet autenticat cap al launcher, especificant-li l'acció a portar a terme
    \item El launcher efectua l'acció, utilitzant la mateixa connexió ja establerta per a comunicar-se
\end{enumerate}

\subsection{REV}

Aquest mode de comunicació, es pot utilitzar només en el cas d'haver llançat el launcher en mode privilegiat.
El nom de REV, prové de la idea principal del mode de comunicació que és l'ús d'una comunicació reversa on 
és el client qui (a través d'un port ja obert en la màquina), sol·licita que el rootkit es connecti cap a ell. \\

El protocol de comunicació és el següent: \\

ESQUEM DEL PROTOCOL \\

\begin{enumerate}
    \item El client obre un servei que es queda escoltant a un port esperant la connexió del launcher.
    \item El client llança un procés fill que es connecta a un port TCP qualsevol de la màquina on s'està 
        executant el rootkit. Aquest port ha d'haver estat obert per qualsevol altre servei del sistema (per 
        exemple el típic servei web).
    \item El client envia un paquet autenticat través d'aquesta connexió. Aquest paquet serà probablement  
        descartat pel servei al no ser un paquet que compleixi el protocol del servei, però serà detectat per
        part del rootkit.
    \item El rootkit detectarà i comprovarà el paquet, i en cas de ser vàlid, establirà una connexió TCP cap 
        al client.
    \item Un cop establerta la connexió amb el client, s'utilitzarà aquesta per tal d'efectuar l'operació 
        demandada.
\end{enumerate}

Els principals avantatges son: \\

\begin{itemize}
    \item La comunicació entre launcher i client és molt fiable i és provable que sobrepassi la majoria de 
        configuracions de xarxa d'una manera totalment vàlida.
    \item Per executar el client, no necessitem permisos de superusuari.
\end{itemize}

Els principals desavantatges són: \\

\begin{itemize}
    \item El primer és que requerim que la màquina disposi d'alguna aplicació que escolti en algun port 
        TCP. Tot i que això no acostuma a ser difícil, hi han casos en què no és així.
    \item El segon és que un cop el rootkit ha establer la connexió TCP amb el client, aquesta connexió
        apareix en el llistat de connexions establertes de la màquina, i en segons quina màquina, això
        pot ser molt sospitós per a l'administrador.
\end{itemize}

Per tal d'utilitzar aquest mode de comunicació, cal que la màquina on s'executa el client, tingui almenys
un port de la ip pública, assignat a ella, de manera que sigui possible lo comunicació directe des de fora
la xarxa local. En configuracions personals com una línia ADSL amb router, caldria que el router de la màquina
on l'executés el client, tingués un ``port obert'' (un port amb DNAT) per tal que el rootkit es pogués 
connectar a ell. Aquest requisit també existeix en els següents modes de comunicació (RAW i LISTEN). \\

\subsection{RAW}

Igual que en el cas anterior, aquest mode de comunicació només pot ser utilitzat en cas d'haver llançat el launcher
en mode privilegiat. \\

Per tal d'implementar aquest mode de comunicació, s'ha hagut d'implementar un protocol de capa de transport 
compatible amb el subset de paquets vàlids pel protocol TCP (documentat més endavant). D'aquesta manera s'ha 
aconseguit poder transmetre per internet paquets TCP vàlids, que a nivell de sessió són aparentment invàlids. 
Com que tots aquests paquets són entregats a la màquina, el nostre rootkit és capaç d'interpretar-los i 
respondre obtenint com a resultat un protocol de comunicació invisible per el nucli del sistema operatiu. \\

Fer tot això, ens aporta principalment dos avantatges: \\

\begin{enumerate}
    \item Que les connexions establertes utilitzant aquest mode, són gairebé invisibles (caldria analitzar els
        diferents paquets de xarxa per detectar una connexió d'aquest tipus)
    \item Que no necessitem tenir cap aplicació escoltant a un port per tal de comunicar-nos amb el launcher.
\end{enumerate}

El nom de mode de comunicació RAW, prové del tipus de socket que ens permet implementar tot això (RAW socket),
i el seu funcionament és el següent: \\

ESQUEM DEL MODE DE COMUNICACIÓ \\

\begin{enumerate}
    \item Primer de tot, el client inicialitza un socket RAW per tal de comunicar-se amb el launcher.
    \item El client crea un procés per tal d'enviar envia el paquet d'autenticació a la màquina i port escollits,
        i espera que el launcher li respongui.
    \item Un cop el launcher rep el paquet, comprova si el paquet és d'alguna connexió existent, i si no
        ho és, crea un altre procés destinat als enviaments de paquets cap al client. Alhora, comença a
        processar l'acció que li ha sol·licitat el client, tot utilitzant els paràmetres rebuts per a la 
        comunicació.
    \item En el moment que el client rep una resposta del launcher, comença a processar l'acció tenint
        ja la connexió establerta.
\end{enumerate}

\subsection{LISTEN}

Aquest mode de comunicació és el què s'utilitza amb el mode de funcionament client, i no requereix de permisos
especials per tal de ser utilitzat. En aquest cas, el client obre un servei per tal d'esperar la connexió del 
launcher.  \\

En el moment en què el client rep la connexió del launcher, aquest li sol·licita una acció en concret a executar
i passa a utilitzar la connexió establerta per a efectuar l'acció. \\

\section{Protocol de comunicació RAW}

FALTA

\section{Paquet de comunicació}

FALTA

ESQUEM DEL PAQUET
