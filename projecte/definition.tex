\chapter{Definició del problema}

Per tal de poder entendre la necessitat de les diferents funcionalitats que ens ha d'oferir el rootkit,
és important que primer de tot, tinguem una idea de com es realitzen avui en dia la gran majoria
d'intrusions\footnote{una intrusió, és l'accés il·legal a una màquina. El fet de introduir-se a una màquina
en temes de seguretat informàtica, també s'acostuma a dir (en un llenguatge més informal) ``hackejar una màquina''},
i el què fan els diferents administradors per protegir-se. Cal tenir en compte que aquests procediments són 
utilitzats alhora d'atacar servidors, en cas de voler atacar pc's, les tècniques utilitzades solen ser diferents.

\section{Intrusions}

Una intrusió és el fet d'aconseguir accés a una màquina normalment remota. D'una manera més tècnica, es podria dir
que l'objectiu d'una intrusió és poder arribar a executar codi\footnote{Amb codi ens referim a bytecode o comandes de sistema} 
a la màquina víctima. \\

La intrusió és el primer pas necessari per tal de poder comprometre del tot una màquina. El conjunt de passos a 
realitzar alhora d'accedir a una màquina es poden resumir com:

\begin{itemize}
\item Intrusió.
\item Augment de privilegis.
\item Instal·lació d'un backdoor.
\item Eliminació de proves.
\end{itemize}

Aquests són els passos a realitzar en una ``bona'' intrusió, tot i que a la majoria de les intrusions que es produeixen
avui en dia, són processos automatitzats que només realitzen el primer i tercer punt.

\subsection{Intrusions automatitzades} 

Avui en dia, la majoria d'intrusions que pateixen els servidors, són intrusions automatitzades causades per cucs o 
programes que exploten un bug o mala configuració en un software en concret. Aquestes intrusions automatitzen les 
tasques a executar un cop s'aconsegueix l'accés a una màquina així com el mateix procediment per aconseguir accedir-hi. 
Per tal de trobar possibles víctimes, aquests programes
es dediquen a escanejar diferents rangs d'IPs en busqueda de màquines que tinguin el servei en qüestió actiu, o 
fan de crawler per tal de trobar noves URLs que indiquin l'ús del software web vulnerable. \\

A una màquina víctima d'una intrusió com aquesta, se li instal·la un tipus de rootkit que permet llançar accions en 
ella (un backdoor). Fins fa poc, la majoria de backdoors d'aquest tipus,  es connectaven a canals de IRC, on 
processaven les comandes que algun altre usuari, llançava. D'aquesta manera, el propietari del backdoor, podia controlar 
d'una manera fàcil, moltes màquines alhora. Per tal de llançar alguna acció en les víctimes, aquest usuari només havia de 
connectar-se al canal en qüestió, i 
enviar un missatge de text, que seria interpretat per totes les màquines que tinguessin el seu backdoor instal·lat. 
Recentment tot això ha començat a canviar una mica, ja que els propietaris d'aquests backdoors estan millorant força la tècnica
per fer-los més efectius. \\

El conjunt de màquines infectades per aquest tipus de backdoors, i que poden ser controlades totes alhora, s'anomena
botnets. Aquestes botnets s'utilitzen principalment per fer atacs massius de tipus DoS contra altres màquines i per enviar SPAM. 
Una dada interessant, és el fet que hi ha gent que s'hi guanya la vida. Per exemple, per una xarxa d'unes 10.000 màquines,
es poden arribar a pagar quantitats entre 15.000 USD i 20.000 USD. \\

El principal canvi que estan realitzant la majoria d¡aquests backdoors, és que estan canviant el protocol de funcionament. Els
més moderns que han aparegut, han canviat l'IRC per l'HTTP i a més, han implementant tècniques de xifratge sobre els diferents 
missatges amb l'objectiu de passar més desapercebuts. \footnote{Un exemple d'això, és la recent notícia de que el servei web 
\href{http://twitter.com/}{Twitter} estava sent utilitzat per a controlar una botnet \url{http://asert.arbornetworks.com/2009/08/twitter-based-botnet-command-channel/}} \\

Les intrusions d'aquest tipus, acostumen a ser provocades per persones que realment no tenen molts coneixements tècnics, sinó que 
agafen algun exploit que s'hagi publicat recentment, i l'intenten modificar per tal de poder fer-ne un atac massiu. El
software utilitzat per gestionar la màquina, no està fet per ells sinó que el descarreguen d'Internet i el configuren. Alhora
aquestes acostumen a ser força joves. \\

Evitar una intrusió d'aquest tipus, acostuma a ser fàcil (tot i que molts cops difícil de portar a terme), ja que 
acostuma a ser suficient en mantenir tot el software de les màquines actualitzat. \\

Detectar quest tipus d'intrusions tampoc acostuma a ser molt difícil. Per detectar-les, acostuma a ser suficient en buscar en el directori /tmp i /var/tmp
per l'existència de fitxers o directoris estranys (ocults, amb espais, etc). També acostuma a ser interessant comprovar 
les connexions de xarxa per detectar si la màquina està registrada en alguna botnet. \\

\subsection{Intrusions manuals}

Les intrusions manuals, depenen molt més de la persona que hi ha darrere la intrusió. \\

La menys perillosa, és aquella que és realitzada per un script kiddie o newbie (el què pretén ser un hacker novell) que 
el què farà serà semblant al propietari de la botnet, però de forma manual. Si la intrusió és satisfactòria, probablement
l'atacant intentarà augmentar els seus privilegis, i deixar algun backdoor per tal de poder mantenir l'accés que ha aconseguit
sense haver de tornar a explotar la vulnerabilitat. Cal tenir en compte que aquestes intrusions acostumen a formar part
de l'aprenentatge de la persona. \\

La intrusió realment perillosa, és la que pot realitzar una persona amb una base tècnica forta. En aquest cas, utilitzar
versions de software en les que no ha estat publicada cap vulnerabilitat, no és suficient (poden haver-hi vulnerabilitats
no conegudes públicament). En aquestes intrusions, l'objectiu sol estar més clar que en els anteriors (ja que 
l'esforç per realitzar-les és molt superior) i pot anar dirigit tant a aplicacions comunes, com a aplicacions fetes a mida. \\

En aquestes intrusions, i depenent de l'objectiu de l'atacant, molt probablement es realitzaran tots els punts comentats anteriorment
per tal de mantenir l'accés i passar desapercebut.  \\

En les intrusions manuals, l'atacant intentarà en tot moment treballar al màxim de còmode.
Això vol dir que si és capaç d'executar shellcode a la màquina que està atacant, no es dedicarà a crear el shellcode
necessari per a cada comanda que vulgui executar al sistema, sinó que tant aviat com pugui cercarà com
aconseguir una shell remota. Aquesta li permetrà d'una manera més o menys còmode, moure's pel sistema i llançar comandes una 
darrere l'altre. \\

Molt probablement en aquesta intrusió, un cop l'atacant hagi acabat, intentarà borrar les dades que demostren que
ha aconseguit l'accés (ex: línies de log del servei que ha explotat).

\section{Mesures de protecció}

Les principals mesures de protecció que es porten a terme per part dels administradors de sistema, per tal d'evitar
intrusions, són les següents:

\begin{description}
    \item[Actualitzacions de seguretat] La principal mesura que ajuda a mantenir els sistemes segurs, és
        realitzar les actualitzacions de seguretat, ja que aquestes corregeixen, totes les vulnerabilitats
        conegudes del software instal·lat a través dels gestors de paquets del sistema operatiu. Cal tenir 
        en compte, que totes les aplicacions externes instal·lades manualment, requereixen de una actualització i 
        comprovació manual. El fet de instal·lar software d'aquesta manera, acostuma a portar forces problemes 
        a la llarga, ja que si l'administrador no està molt al corrent del programari que instal·la,
        quan apareixen vulnerabilitats, li passen desapercebudes i a la màquina hi acaba quedant un software 
        vulnerable i mig oblidat.
    \item[Firewall d'entrada] La segona mesura que s'acostuma a aplicar a la vida real, és el fet de configurar
        un firewall que només permeti accedir als serveis que realment s'estan oferint a la màquina, de manera
        que si algun servei que està instal·lat a la màquina no ha de ser accessible des de fora, el firewall
        no hi permet l'accés. Això evita en el cas d'una intrusió, el poder deixar un servei escoltant a un port
        de la màquina de tal manera que més endavant es pugui accedir a ella.
    \item[Firewall de sortida] Un firewall de sortida, ja no és una pràctica tant comuna. Només els ISPs més 
        professionals (i en definitiva, els més conscienciats amb la seguretat) són els què acostumen a aplicar 
        polítiques de denegació del trànsit de sortida. El fet d'aplicar aquest tipus de polítiques, realment
        evita moltes possibles intrusions. Per exemple, un ISP amb aquestes polítiques però amb aplicacions 
        vulnerables instal·lades, probablement evitaria el que un possible propietari d'una botnet que utilitzés
        el IRC per controlar les màquines, s'apoderés de les seves. En aquest exemple, l'exploit explotaria 
        correctament l'aplicació vulnerable, però a l'hora de intentar connectar-se al canal de IRC per tal
        d'anunciar que està infectada, i esperar ordres, mai ho arribaria a aconseguir ja que la connexió 
        seria denegada per el firewall. En els casos on s'utilitza un firewall de sortida, aquest no acostuma
        a denegar tot el transit, sinó que permet resolucions DNS, trànsit local, etc, i molts cops trànsit web. 
        És per aquest motiu, que les botnets estan canviant al protocol HTTP per a controlar 
        les seves màquines en comptes IRC (que era el més utilitzat fins ara).
    \item[Restriccions de sistema operatiu] A part dels punts comentats anteriorment, existeixen peces de 
        software que intenten restringir la pròpia explotació de vulnerabilitats tot endurint la seguretat que 
        envolta les aplicacions. Programari com SELinux, opcions de compilació del gcc, randomització del kernel, 
        etc, són altres tècniques que s'acostumen a utilitzar, però d'una manera més involuntària ja que és 
        el propi SO el què et proveeix d'aquestes funcionalitats, i en cas de no fer-ho, és molt rara la vegada
        que una protecció d'aquestes és afegida a un sistema que no la porta ja incorporada.
    \item[Scripts personalitzats de comprovació] Una altre mesura també utilitzada, és la creació d'algun script per tal de 
        comprovar si hi ha algun procés o fitxer sospitós. Aquesta mesura no és tant de protecció, sinó que 
        permet a l'administrador rebre una alarma en un cas concret. D'aquesta manera, l'administrador és possible
        que pugui actuar més o menys a temps. Un script típic, és el què busca execucions d'una shell que pengen
        d'un procés de servidor web.
\end{description}

\section{Plans de contingència}

A part d'aquestes mesures de seguretat que permeten evitar molts incidents, els administradors de sistemes un cop
sospiten que han patit alguna intrusió, intentaran descobrir fins a on ha arribat la intrusió, què han instal·lat 
per a mantenir l'accés, i què han fet exactament. Per comprovar tot això, existeixen diferents utilitats que busquen si en 
el sistema hi han aplicacions sospitoses instal·lades. El problema que tenen aquestes utilitats, és que només detecten
aplicacions genèriques, o aplicacions que utilitzen les tècniques més comunes. Exemples d'aquestes utilitats serien 
rkthunter, checkrootkit, etc. \\

En cas que aquestes utilitats no detectin res (serien com una espècie d'antivirus), poden causar a l'administrador
una falsa sensació de seguretat. Si tot i això l'administrador detecta el rootkit, ell mateix intentarà descobrir 
què és el que fa. És en aquest punt on igual  que els virus, les tècniques d'ocultació i antidebug, prenen sentit.
Si l'administrador no és capaç de descobrir el què fa el rootkit, és molt possible que no el pugui eliminar de cap
altre manera que no sigui reinstal·lant la màquina. \\

\section{Objectiu}

Un cop tenim una visió més clara del què vol un atacant en el moment que realitza una intrusió, podem entendre 
més fàcilment els objectius a grans trets que tindrà aquest rootkit, que principalment seran:

\begin{itemize}
    \item Permetre l'accés i el nivell de privilegis a la màquina
    \item Ocultar-se
    \item Oferir un entorn el màxim de còmode
    \item Evitar que el depurin i descobreixin el seu funcionament
    \item Compatible amb la majoria de màquines
\end{itemize}

En el nostre projecte, comencem a treballar a partir del moment en què ja hem aconseguit l'accés a la
màquina. En el moment en què ja som capaços d'executar codi, és quan el nostre rootkit, volem que ens 
comenci a ser útil. \\
