\chapter{Glossari}

\nomenclature[GNU]{GNU s Not Unix}{A computer operating system composed entirely of free software.}
\nomenclature[linux]{Linux}{Any Unix-like computer operating system that uses the Linux kernel.}
\nomenclature[BSD]{Berkeley Software Distribution}{Sistema operatiu de la universitat de Berkeley. Sistema basat en UNIX.}
\nomenclature[malware]{Malware}{Peça de software que té un objectiu maliciós.}
\nomenclature[virus]{Virus}{Un virus informático es un malware que té per objectiu alterar el funcionament de la màquina.}
\nomenclature[cavall de troia]{Cavall de troia}{}
\nomenclature[rootkit]{Rootkit}{Èina o grup d'èines que té com a finalitat 
amagar-se a ella mateixa, i amagar altres programaes, processos, arxius, directoris, ports, etc., 
per tal que permeti a un intrús accedir al sistema principalment remotament, així com extreure informació.}
\nomenclature[TTY]{TTY}{Un tty, és una utilitat de UNIX que ens permet sessions molt més interactives a un terminal.}
\nomenclature[ELF]{Executable and Linking Format}{Aquest és el format per defecte dels fitxers executables i llibreries.}
\nomenclature[socket]{socket}{Interfície del sistema utilitzada principalment per a les comunicacions a través de la xarxa.}
\nomenclature[POSIX]{POSIX}{}
\nomenclature[keylogger]{Keylogger}{}
\nomenclature[pty]{pseudotty}{}
\nomenclature[shell]{shell}{}
\nomenclature[portable]{Codi/aplicació portable}{}
\printnomenclature

