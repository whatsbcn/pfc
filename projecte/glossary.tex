
%\usepackage{syntonly}
%\syntaxonly
%\usepackage[style=list,toc=true]{glossary}
%\makeglossary
%\newacronym[PURL]{PURL}{persistent uniform resource locator}{description={A means of creating persistent links to 
%networked resources through the use of an intermediate resolution service.  The PURLs used in this thesis are 
%provided by the Online Computer Library Center (\url{http://purl.org}).}}
%\newacronym[PURL2]{GNU}{GNU s Not Unix}{description={A computer operating system composed entirely of free software.}}
%\newacronym{linux}{Linux}{description={Any Unix-like computer operating system that uses the Linux kernel.}}
%\newacronym{BSD}{Berkeley Software Distribution}{description={Sistema operatiu de la universitat de Berkeley. Sistema basat en UNIX.}}
%\newacronym{virus}{Virus}{description={Un virus informático es un malware que té per objectiu alterar el funcionament de la màquina.}}
%\newacronym{cavall de troia}{Cavall de troia}{description={}}
%\newacronym{malware}{Malware}{description={Peça de software que té un objectiu maliciós.}}
%\newacronym{rootkit}{Rootkit}{description={Èina o grup d'èines que té com a finalitat 
%amagar-se a ella mateixa, i amagar altres programaes, processos, arxius, directoris, ports, etc., 
%per tal que permeti a un intrús accedir al sistema principalment remotament, així com extreure informació.}}
%\makeindex
