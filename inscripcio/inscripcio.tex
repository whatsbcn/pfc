% Tipus de document
\documentclass[a4paper]{article}
% Extres de LaTex
\usepackage{latexsym}
% Text en utf8
\usepackage{ucs}
\usepackage[utf8x]{inputenc}
% Text generat automàticament en català i partició per síl·labes
\usepackage[catalan]{babel}


% Autor
\author{Albert Sellarès}
% Títol de l'article
\title{skd: Rootkit per sistemes operatius UNIX}

% Millora el pdf. Ha de ser l'últim del preàmbul
\usepackage[pdftex,bookmarks,colorlinks,pdfnewwindow]{hyperref}
\hypersetup{linkcolor=black}

\begin{document}

% Portada
\maketitle
\newpage


\section{Introducció}

Un rootkit és una aplicació pensada per a ser utilitzada com a porta del darrere per tal de
poder accedir i controlar un sistema, i que a més, s'amaga per a no ser descobert.
És sabut que existeixen programes fets amb aquesta finalitat, però la majoria d'ells són
per sistemes Windows i no estan públics a la xarxa, o estan desfasats i ja no poden ser
usats en els sistemes operatius actuals.
A més, els nuclis de sistemes operatius actuals, implementen cada vegada més
proteccions per tal d'evitar que programes com aquests els controlin.
Aquest projecte tracta d'intentar crear una prova de concepte del què seria un rootkit
avançat per sistemes basats en UNIX, tenint gran pes en ell tota la part de recerca i
investigació.

\section{Motivació del projecte}

Avui en dia ens veiem immersos en la societat de la informació, un moment en què la
majoria de les empreses i persones intentem adaptar-nos a les noves tecnologies,
moment en què s'intenta digitalitzar tot el què es pot.
Qui més qui menys veu que en uns anys tot es veurà gestionat a través de sistemes
d'informació, tot estarà interconnectat entre sí, i s'ha de poder garantir que tots aquests
sistemes, comptin amb un mínim de seguretat.
Des de els inicis de la història els sistemes operatius UNIX han estat al capdavant en els
entorns de servidors, i des de llavors que les grans empreses els utilitzen per a confiar-hi
les seves dades. Avui, i gràcies al boom que ha fet el GNU/Linux (tothom qui més qui
menys té una idea del què és), administradors de sistemes no experimentats acaben
utilitzant sistemes basats en UNIX en el seu lloc de treball.
Les implicacions que té estar infectat per un virus estan canviant. Els virus (i en particular
els cavalls de troya) ja no es fan per molestar a l'usuari, al contrari, un alt percentatge dels
virus tenen com a objectiu principal, passar desapercebuts per tal de permetre l'infectant
controlar la nostre màquina, podent fer qualsevol cosa en ella com per exemple, apoderar-
se de la nostra compta bancària.
Parem-nos a pensar per un moment què pot passar si en comptes de què l'infecció estigui
a la nostre màquina, aquesta estigui als servidors on es fan les nostres nòmines, als dels
nostre banc, o a les de qualsevol pàgina de venta d'articles per internet. El perill i la
criticitat, es disparen exponencialment.
Són moltes les empreses que utilitzen sistemes operatius basats en UNIX (com poden ser
GNU/Linux o BSD) per als seus servidors, i són moltes les què s'acaben despreocupant

de la seguretat dels servidors donant com a excusa que un sistema basat amb UNIX és
més segur, i que a més, no hi han virus. A la practica, molta gent que es dedica a
administrar aquestes màquines, no està qualificada, o no compta amb el temps necessari
per a fer-ho del tot bé, i com que les coses aparentment funcionen, es segueix així.
I aquí és on vull arribar, avui en dia una persona amb suficients ganes, temps i
coneixements, pot guanyar accés a la majoria de servidors, on un cop accedit, la seva
preocupació principal, és mantenir-ne l'accés.
Amb aquest projecte, intento posar de manifest creant una prova del concepte, lo difícil
que pot arribar a ser per un administrador de sistemes adonar-se que ha estat infectat per
un rootkit, lo perillós per a la seguretat del sistema, i lo crua que és la realitat ja que un
intrús pot tenir-ho molt fàcil per a controlar el nostre servidor.

\section{Objectiu del projecte}

L'objectiu d'aquest projecte és implementar un rootkit per a sistemes operatius actuals
basats en UNIX.
L'objectiu del projecte, és implementar un rootkit per a sistemes basats en UNIX. La idea
és potenciar al màxim les seves característiques que com a tot rootkit, són les seves
següents:

Ocultació. El rootkit ha de passar el màxim desapercebut en el sistema que ha
estat instal·lat. Per un administrador ha de ser molt difícil d'adonar-se que el
sistema ha estat compromès. El rootkit ha d'intentar ocultar al màxim les tasques
que executa.
Permetre accés. El rootkit ha de permetre l'accés a la màquina a la persona que
l'ha instal·lat. Normalment i per comoditat, aquest accés és remot a través
d'internet, havent d'evitar les diferents barreres que hi puguin haver.
Administració remota de la màquina. El rootkit ha de permetre realitzar tot tipus
d'accions a la màquina com si d'un usuari vàlid es tractés. Accions com executar
tasques, editar fitxers, pujar o descarregar-ne, són funcionalitats bàsiques.
Permanència. Les màquines s'actualitzen, es paren i s'engeguen. El rootkit ha
d'intentar no veure's afectat per aquests canvis.
Augment de privilegis. Ha de ajudar tant com pugui, a la obtenció de nous
privilegis.

%\gls{linux} % displays name field of the linux entry (in this case "Linux")
%\useGlosentry{linux}{GNU/Linux} % displays "GNU/Linux"
%\GNU % displays "GNU's Not Unix (GNU)" the first time this is used
%\GNU % displays "GNU" all subsequent times
% NB: remember to use \GNU\ if want to retain the space after the acronym

%\section{Glossary}
%\printglossaries
%\printglossary
%\addcontentsline{toc}{chapter}{Glossary} % remove this line if you don't want a table of contents entry for the glossary
%\newpage

%\appendix
%test
%ae
%ts
%et
%sdt
%Bibliografia wikipedia
%\include{filename}

\end{document}
