\renewcommand{\nomname}{Glossari}

%<*sample05>
%\def\@@@nomenclature[#1]#2#3{%
%\def\@tempa{#2}\def\@tempb{#3}%
%\protected@write\@nomenclaturefile{}%
%	{\string\nomenclatureentry{#1\nom@verb\@tempa @[{\nom@verb\@tempa}]%
%		|nompageref{\begingroup\nom@verb\@tempb\protect\nomeqref{\theequation}}}%
%	{\thepage}}%
%\endgroup
%\@esphack}

%\def\nompageref#1#2{%
%	\if@printpageref\pagedeclaration{#2}\else\null\fi
%	\linebreak#1\nomentryend\endgroup}

%\def\pagedeclaration#1{\dotfill\nobreakspace#1}
%\def\nomentryend{.}
\def\nomlabel#1{\textbf{#1}\hfil}

\nomenclature[rootkit]{Rootkit}{Eina o grup d'eines que té com a finalitat 
    amagar-se a ella mateixa, i amagar altres programames, processos, arxius, directoris, ports, etc., 
    per tal que permeti a un intrús accedir al sistema, remotament o no, així com extreure informació.}

\nomenclature[backdoor]{Backdoor}{Veure definició de Porta del darrere.}

\nomenclature[porta del darrere]{Porta del darrere}{És un mètode per sobrepassar el sistema d'autenticació normal d'una 
    màquina remota. La funció de porta del darrere la pot fer tant un software afegit a la màquina, com la modificació
    d'un software existent. En anglès s'anomena backdoor}

\nomenclature[nucli de sistema operatiu]{Nucli del sistema operatiu}{El núcli del sistema operatiu o kernel, és el component
    central dels sistemes operatius. La seva principal responsabilitat és la gestió dels recursos del sistema de manera 
    justa i segura.}

\nomenclature[kernel]{Kernel}{El kernel és el nucli del sistema operatiu.}

\nomenclature[virus]{Virus}{Un virus informàtic que té per objectiu alterar el funcionament de la màquina per a realitzar
    tasques que el propietari de la màquina no ha sol·licitat, ni vol.}

\nomenclature[cavall de troia]{Cavall de Troia}{Tipus de virus informàtic que permet que externament es pugui accedir a les 
    dades del sistema.}

\nomenclature[UNIX]{Unix}{Sistema operatiu creat el 1969 per l'empresa AT\&T Bell a partir del qual, molts altres sistemes 
    operatius van basar-se en la seva arquitectura i principis.}

\nomenclature[malware]{Malware}{Peça de software que té un objectiu maliciós.}

\nomenclature[programari lliure]{Programari Lliure}{El programari lliure (en anglès free software) és el programari que pot 
    ser usat, estudiat i modificat sense restriccions, i que pot ser copiat i redistribuït bé en una versió modificada o 
    sense modificar sense cap restricció, o bé amb unes restriccions mínimes per garantir que els futurs destinataris també 
    tindran aquests drets.}

\nomenclature[GNU]{GNU s Not Unix}{Projecte que va néixer amb l'objectiu de crear un sistema operatiu del tot lliure.}

\nomenclature[linux]{Linux}{Nucli del sistema operatiu GNU/Linux. Aquest està basat en Minix, i utilitza una llicència lliure
    per distribuir-se.}

\nomenclature[BSD]{Berkeley Software Distribution}{Sistema operatiu de la universitat de Berkeley inicialment basat en UNIX.}

\nomenclature[solaris]{Solaris}{Solaris és un sistema operatiu de l'empresa Sun Microsystems inicialment basat en el sistema UNIX.}

\nomenclature[hacker]{Hacker}{Persona entusiasta amb el què fa, que té una gran passió en adquirir més coneixement per entendre
    amb detall com funcionen les coses, i en definitiva per aprendre. Aquest terme és utilitzat principalment en temes de seguretat informàtica.}

\nomenclature[cracker]{Cracker}{Persona que té per objectiu realitzar accions perjudicials per algú. En la majoria de mitjans de comunicació 
    quan s'utilitza la paraula hacker en realitat hauria d'utilitzar-se cracker.}

\nomenclature[hackejar]{Hackejar un sistema}{El fet d'introduir-se a un sistema anomenat d'una manera més col·loquial. També
    s'anomena així el fet d'estudiar a fons un sistema.}

\nomenclature[intrusio]{Intrusió}{Una intrusió és l'accés il·legal a una màquina.}

\nomenclature[crawler]{Crawler}{És un programa que navega per el World Wide Web en busca d'alguna cosa.}

\nomenclature[botnet]{Botnet}{Una botnet és una xarxa de màquines que treballa de forma automatitzada i autònoma. Aquesta pot 
    ser controlada o totalment automàtica.}

\nomenclature[dos]{DoS}{Tipus d'atac anomenat Denegació de Servei, que té per objectiu denegar l'ús d'un servei per a la majoria
    dels seus usuaris.}

\nomenclature[exploit]{Exploit}{Peça de software que s'ha desenvolupat per tal d'explotar alguna vulnerabilitat.}

\nomenclature[script kiddie]{Script Kiddie}{Persona que en el món de la seguretat informàtica, només es dedica a descarregar i
    executar exploits desenvolupats per tercers per atacar sistemes. Són usuaris amb uns coneixements molt limitats.}

\nomenclature[newbie]{NewBie}{És tota persona que està aprenent alguna cosa. S'anomenen així els ``Aprenents de hacker''.}

\nomenclature[shellcode]{Shellcode}{Conjunt de bytes que interpretat per un processador, són executats com a operacions vàlides.
    Molts exploits acostumen a incorporar un shellcode que és el codi que s'executarà un cop s'hagi compromès la màquina.}

\nomenclature[shell]{Shell}{Una shell és un intèrpret de comandes del món UNIX.}

\nomenclature[shell remota]{Shell Remota}{És un intèrpret de comandes, però que pot ser usat remotament (a través d'una xarxa o internet).}

\nomenclature[log]{Log}{Fitxer de registre on es van registrant diferents missatges per tal de tenir constància dels diferents events que 
    va llançant un servei en concret.}

\nomenclature[firewall]{Firewall}{Software o hardware que es dedica a filtrar els paquets de xarxa per tal de modificar-los, permetre'ls 
    o denegar-los.}

\nomenclature[resolucio DNS]{Resolució DNS}{Fet de seguir el protocol DNS per obtenir una direcció IP a partir d'un nom.}

\nomenclature[HTTP]{HTTP}{HyperText Transfer Protocol. Protocol utilitzat en el món web per tal de transferir la informació entre client i servidor.}

\nomenclature[IRC]{IRC}{Internet Relay Chat. Protocol utilitzat per a la comunicació de persones a través del què s'anomenen sales de Xat.}

\nomenclature[selinux]{SELinux}{Security-Enhanced Linux. És una funcionalitat de les versions més recents de linux que permet definir
    una sèrie de polítiques per enfortir la seguretat dels sistemes.}

\nomenclature[gcc]{Gcc}{Gnu C Compiler. Compilador de C de GNU. Un dels projectes més importants de GNU}

\nomenclature[randomitzacio de memoria]{Randomització de memòria}{Tècnica utilitzada en els sistemes operatius actuals que tracta 
    de generar les direccions de memòria de forma aleatòria en cada execució per tel de fer-ne més difícil la seva explotació en cas d'errada.}

\nomenclature[antidebug]{Antidebug}{Les tècniques antidebug són totes aquelles tècniques utilitzades per fer més difícil la depuració 
    i anàlisis d'un executable.}

\nomenclature[portable]{Codi/aplicació portable}{Aquell/a que permet ser portat/compilat/executat en més d'un tipus de sistema, sense requerir
    de canvis.}

\nomenclature[root]{Root}{Usuari administrador dels sistemes Unix.}

\nomenclature[TTY]{TTY}{Un tty és una utilitat de UNIX que ens permet emular sessions interactives a un terminal.}

\nomenclature[ELF]{Executable and Linking Format}{Aquest és el format per defecte dels fitxers executables i llibreries en els sistemes Unix actuals.}

\nomenclature[socket]{Socket}{Interfície del sistema utilitzada principalment per a les comunicacions a través de la xarxa.}

\nomenclature[POSIX]{POSIX}{Portable Operating System Interface. Interfície que estandaritza l'API de sistema dels sistemes Unix.}

\nomenclature[keylogger]{Keylogger}{Utilitat que té per objectiu capturar els caràcters escrits per l'usuari, normalment amb l'afany d'aconseguir 
    contrasenyes o dades secretes.}

\nomenclature[bytecode]{Bytecode}{És un sinònim de shellcode o codi màquina. Aquesta paraula prové d'unir la paraula byte més code. És així ja que el bytecode és
    una tira de bytes que és executada com a codi màquina.}

\nomenclature[escanejar]{Escanejar}{S'anomena escanejar una xàrxa o una màquina, el fet de anar passant per tots els components en busca d'alguna cosa. Per ejemple
    quan diem escanejar una màquina, ens podem referir escanejar en busca de vulnerabilitats.}

\nomenclature[spam]{SPAM}{És la publicitat que ens és entregada sense haver-la sol·licitat. Típicament la publicitat que rebem al correu electrònic.}

\nomenclature[NAT]{Setwork Address Translation}{És una tècnica que s'utilitza avui en dia per poder reutilitzar una sola ip pública per a 
    més d'una màquina. Tracta d'enmascarar la ip de les màquines en qüestió, tot substituint-la per la del dispositiu que realitza el NAT. Aquestà tècnica
    és molt utilitzada avui en dia. És la que realitzen la majoria de router ADSL dels internautes de l'estat Espanyol.}

\nomenclature[opensource]{Open Source}{És considerat opensource tot aquell software del qual el seu codi font és públic.}

\nomenclature[violaciodesegment]{Violació de segment}{Error que es produeix quan l'execució 
d'un programa, provoca (normalment a causa d'un bug) que aquest accedeixi a un segment de memòria que no li pertany.
En intentar-ho, el sistema operatiu ho evita, i llança aquest error.}

\printnomenclature
\addcontentsline{toc}{chapter}{11 Glossari} 
