\clearpage
\phantomsection
\addcontentsline{toc}{chapter}{\numberline {10}Bibliografia}

\begin{thebibliography}{9}
 
\bibitem{dietlibc}
  Felix von Leitner,
  \emph{Diet libc, a libc optimized for small size}.
  http://www.fefe.de/dietlibc/,
  2009.
 
\bibitem{glibc}
  Free Software Foundation, Inc,
  \emph{GNU C library}.
  http://www.gnu.org/software/libc/,
  2001-2009.

\bibitem{unix-programming}
  W. Richard Stevens, Stephen A. Rago,
  \emph{Advanced Programming in the UNIX Environment, Second Edition}.
  ISBN 0321525949,
  2008.


\bibitem{milw0rm}
  Str0ke (str0ke[at]milw0rm.com),
  \emph{milw0rm - exploits : vulnerabilities : videos : papers : shellcode}.
  \url{http://www.milw0rm.com/},
  2003-2009.

\bibitem{suckit rootkit}
	sd (sd@sf.cz) and devik (devik@cdi.cz),
	\emph{Linux on-the-fly kernel patching without LKM},
	\url{http://www.phrack.org/issues.html?issue=58&id=7&mode=txt},
	2001.

\bibitem{dash}
	\emph{DASH is a direct descendant of the NetBSD POSIX version of ash}.
	\url{http://gondor.apana.org.au/~herbert/dash/},
	2002-2009.

\bibitem{openssh}
	\emph{OpenSSH is a FREE version of the SSH connectivity tools}.
	\url{http://www.openssh.com/},
	1999-2009.

\bibitem{socks4}
	NEC Global Gateway Corporation,
	\emph{Especificacions formals del protocol SOCKS4 i SOCKS4a}.
	\url{http://ftp.icm.edu.pl/packages/socks/socks4/SOCKS4.protocol}

\bibitem{chkrootkit}
	\emph{Chkrootkit is a tool to locally check for signs ora rootkit}.
	\url{http://www.chkrootkit.org/}

\bibitem{rkhunter}
	\emph{Rkhunter is a scanner for known and unknown rootkits, backdoors, and sniffers}.
	\url{http://www.rootkit.nl/},
	2003-2009

\bibitem{wikipedia}
	\emph{Wikipedia, the free encyclopedia}.
	\url{http://en.wikipedia.org/wiki/},
	2001-2009.

\end{thebibliography}
